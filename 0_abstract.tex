Die Nutzung von Cloud-Diensten wird immer beliebter, da sie große Datenmengen effizient speichern, schnellen Zugriff auf Ressourcen bieten und einen nahtlosen Datenaustausch ermöglichen. Allerdings steigt damit auch das Risiko, dass Daten in der Cloud ungewollt offengelegt werden. Das Problem der Datenverluste ist zu einer kritischen Herausforderung geworden, die die Entwicklung umfassender Systeme in diesem Bereich erforderlich macht. Ein System zur Verhinderung von Datenlecks (Data Leakage Prevention, DLP) konzentriert sich darauf, sensible Daten zu identifizieren und klassifizieren und Verstöße gegen Sicherheitsrichtlinien zu verhindern und zu melden. Ein wesentlicher Aspekt dieses Systems ist die automatische Klassifizierung von sensiblen Daten. Es gibt zwar manuelle Methoden, aber maschinelle Lernverfahren haben sich als äußerst effektiv erwiesen. Cloud-DLP-Systeme können auf der Grundlage der klassifizierten Daten verschiedene Sicherheitsmaßnahmen ergreifen und so den Schutz vor Datenlecks verbessern.

