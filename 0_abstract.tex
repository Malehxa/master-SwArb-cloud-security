Die Cloud wird aufgrund ihrer Flexibilität, Skalierbarkeit und kostengünstigen Ressourcenverwaltung immer beliebter. Mit dem Einsatz von Cloud-Diensten steigt jedoch auch die Komplexität und die Verantwortung liegt beim Nutzer. Diese Faktoren können in Unternehmen zu unbeabsichtigten Datenlecks führen. Der Einsatz von Cloud-Data-Leakage-Prevention-Systemen zur Verhinderung von Datenlecks nimmt somit an Bedeutung zu. Sie konzentrieren sich darauf, sensible Daten zu identifizieren und klassifizieren und Verstöße gegen Sicherheitsrichtlinien zu verhindern und zu melden. Ein wesentlicher Aspekt dieser Systeme ist die automatische Klassifizierung von sensiblen Daten. Durch die zunehmenden Datenmengen im Kontext von Big Data sind hier effiziente Technologien und Methoden erforderlich. Der Fortschritt im Bereich künstlicher Intelligenz (KI) bietet hierbei einen vielversprechenden Ansatz. KI-basierte Methoden zur automatischen Datenklassifizierung können lernen, sensible Informationen zu erkennen und zu klassifizieren. Die Verwendung von Methoden wie k-NN, Boosting, Clusteranalyse und NLP-Modelle haben dabei gute Ergebnisse erzielt. Cloud-DataLeakage-Prevention-Systeme können dann auf Basis der klassifizierten Daten verschiedene Sicherheitsmaßnahmen ergreifen und so den Schutz vor unbeabsichtigten Datenlecks verbessern.

