Die Cloud wird aufgrund ihrer Flexibilität, Skalierbarkeit und kostengünstigen Ressourcenverwaltung immer beliebter. Mit dem Einsatz von Cloud-Diensten steigt jedoch auch die Komplexität. Zudem verschiebt sich die Verantwortung zum Nutzer. Diese Faktoren können in Unternehmen zu unbeabsichtigten Datenlecks führen. Der Einsatz von Systemen zur Verhinderung von Datenlecks in der Cloud wird immer wichtiger, um Datenschutzverletzungen in der Cloud zu verhindern. Diese Systeme sind darauf ausgelegt, sensible Daten zu erkennen und zu klassifizieren sowie Verstöße gegen Sicherheitsrichtlinien zu verhindern und zu melden. Ein wesentlicher Aspekt dabei ist die automatische Klassifizierung von sensiblen Daten. Durch die zunehmenden Datenmengen im Kontext von Big Data sind effiziente Technologien und Methoden erforderlich. Der Fortschritt im Bereich künstlicher Intelligenz (KI) bietet hierbei einen vielversprechenden Ansatz. KI-basierte Methoden zur automatischen Datenklassifizierung können lernen, sensible Informationen zu erkennen und zu klassifizieren. Die Verwendung von Methoden wie k-NN, Boosting, Clusteranalyse und Sprachverarbeitungsmodelle haben dabei gute Ergebnisse erzielt. Cloud-Data-Leakage-Prevention-Systeme können auf Basis der klassifizierten Daten verschiedene Sicherheitsmaßnahmen ergreifen und so den Schutz vor unbeabsichtigten Datenlecks verbessern.

