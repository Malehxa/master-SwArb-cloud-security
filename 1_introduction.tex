% Einleitung warum Cloud immer beliebter wird
Im Jahr 2022 gaben bereits 84\% der befragten 552 Unternehmen in Deutschland an, dass sie Cloud-Dienste in ihrem Unternehmen einsetzen \cite{KPMG.2022}. Cloud Computing hat sich in der Zeit der digitalen Transformation zu einem wichtigen Bestandteil der Informationsverarbeitung entwickelt. Die Nutzung von Cloud-Diensten wird immer beliebter, da sie die Möglichkeit zur effizienten Speicherung großer Datenmengen, schnellen Zugang zu Ressourcen und nahtlosen Datenaustausch bietet. Durch den Verbreitung von digitaler Technologie in der Gesellschaft und in Unternehmen werden immer mehr Daten geteilt und gesammelt. Um diese großen Datenmengen sammeln und verarbeiten zu können, nutzen Unternehmen die Vorteile von Cloud-Diensten. Die Möglichkeit, Daten in Echtzeit zu teilen, verbessert Geschäftsprozesse und erleichtert die Zusammenarbeit im Unternehmen \cite{Surianarayanan.2023b}. %18-27

% warum Datensicherheit
Da Informationen das wertvollste Gut eines Unternehmens sind, ist ihr Schutz von größter Bedeutung. Beim Sammeln von Daten ist ein Unternehmen zudem verpflichtet, sie vor Diebstahl, Verlust und Missbrauch zu schützen. Es gibt zahlreiche Datenschutzgesetze und -vorschriften, wie die EU-Datenschutz-Grundverordnung (DSGVO), um sensible Daten wie personenbezogene Daten zu schützen. Ziel dieser Vorschriften ist es, strengen Regeln für das Sammeln von Daten vorzugeben und der Einzelperson eine vergleichsweise hohe Kontrolle über ihre personenbezogenen Daten zu geben \cite{Kuzina.2023}. % 1
Unabhängig des Speicherorts besteht also das Risiko, dass die Datensicherheit verletzt wird.

% Relevanz Data Leakage Prevention
Eines der Hauptziele der Informationssicherheit ist die Verhinderung der Offenlegung von Daten gegenüber Unbefugten. Datenlecks können jedoch aufgrund der Notwendigkeit, auf Informationen zuzugreifen, diese zu teilen und zu nutzen, nicht immer verhindert werden. Diese Bedrohung kann von böswilligen Außenstehenden ausgehen, die versuchen, sensible Daten zu erhalten. Umgekehrt können auch interne Mitarbeiter eine Gefahr darstellen, wenn sie beabsichtigt oder unbeabsichtigt Informationen preisgeben \cite{Alneyadi.2016}. Bereits im Jahr 2018 haben Studien gezeigt, dass 53\% der befragten Unternehmen Insider-Angriffe in den letzten 12 Monaten bestätigten. Dabei sind Bedrohungen von innen häufig schwerwiegender als von außen, da sie meist schwieriger zu erkennen sind \cite{CATechnologies.2018}.
Die Offenlegung von sensiblen Daten kann erheblichen Schaden verursachen. Unternehmen können ihren Wettbewerbsvorteil verlieren, ihr Image beeinträchtigen, Umsatzeinbußen erleiden oder sogar Geldstrafen und Sanktionen erhalten.

Um das Risiko von Datenschutzverletzungen zu minimieren, werden immer häufiger Data-Leakage-Prevention (DLP) Lösungen eingesetzt. Gartner prognostiziert, dass bis 2027 etwa 70\% der größeren Unternehmen eine DLP-Lösung einsetzen werden, um die Datensicherheit vor Insider-Risiken und externen Angreifern zu schützen \cite{Chugh.2023}. DLP-Systeme überwachen den Zugriff und Austausch vertraulicher Daten, um unbefugte Offenlegung oder missbräuchliche Nutzung zu erkennen.

% Fortschritt in KI hilft auch Cloud Security
Unternehmen sammeln häufig große Datenmengen, ohne zu wissen, was erfasst wird oder wie sie nach personenbezogenen Daten suchen oder diese abrufen können. Das erschwert den Schutz der Privatsphäre. DLP-Systeme benötigen die Information, ob bestimmte Daten besonders schützenswert sind oder nicht. Im Zeitalter von Big Data ist es jedoch kaum noch möglich, die enormen Datenmengen manuell zu analysieren. Der Fortschritt im Bereich künstliche Intelligenz (KI) bietet hierbei einen vielversprechenden Ansatz. KI-basierte Methoden zur automatischen Datenklassifizierung können in DLP-Systemen eingesetzt werden, um sensible Informationen zu erkennen.

% Fokus der Arbeit, Struktur
Aufgrund der neuen Möglichkeiten mit dem Einsatz von KI im Bereich Datenschutz liegt der Fokus in dieser Arbeit auf der Anwendung von Methoden der automatischen Datenklassifizierung zur Erkennung sensibler Informationen, um den Schutz sensibler Daten zu gewährleisten. Diese Arbeit beschäftigt sich mit der Frage, wie sensible Daten in große Datenmengen am besten erkannt werden können. Dabei wird zunächst die Bedrohung durch versehentliche Offenlegung von Daten beschrieben und anschließend die Abwehrmaßnahme 'Data Leakage Prevention' vorgestellt. Dabei liegt der Fokus auf der Erkennung von sensiblen Informationen. Es werden verschiedene KI-basierte Methoden und ihre Funktionsweise im Bezug auf Datenklassifizierung vorgestellt. Anschließend wird deren Einsatz in der Cloud Sicherheit diskutiert.