% Einleitung warum Cloud immer beliebter wird
Im Jahr 2022 gaben bereits 84\% der befragten 552 Unternehmen in Deutschland an, Cloud-Dienste zu verwenden \cite{KPMG.2022}. Cloud Computing hat sich während der digitalen Transformation zu einem wichtigen Bestandteil der Informationsverarbeitung entwickelt. Die Beliebtheit von Cloud-Diensten nimmt stetig zu, da sie die effiziente Speicherung großer Datenmengen, den schnellen Zugang zu Ressourcen und den nahtlosen Datenaustausch ermöglichen. Mit der Verbreitung digitaler Technologien in der Gesellschaft und in Unternehmen werden zunehmend mehr Daten geteilt. Unternehmen nutzen die Vorteile von Cloud-Diensten, um diese umfangreichen Datenmengen zu sammeln und zu verarbeiten. Die Möglichkeit, Daten in Echtzeit zu teilen, verbessert Geschäftsprozesse und erleichtert die unternehmensinterne Zusammenarbeit \cite{Surianarayanan.2023b}. %18-27

% warum Datensicherheit
Da Informationen das wertvollste Gut eines Unternehmens darstellen, ist ihr Schutz von größter Bedeutung. Beim Sammeln von Daten ist ein Unternehmen zudem verpflichtet, diese vor Diebstahl, Verlust und Missbrauch zu schützen. Zahlreiche Datenschutzgesetze und -vorschriften, wie die EU-Datenschutz-Grundverordnung (DSGVO), wurden eingeführt, um sensible Daten wie personenbezogene Informationen zu sichern. Ziel dieser Vorschriften ist es, klare Richtlinien für das Sammeln von Daten festzulegen und Einzelpersonen eine vergleichsweise hohe Kontrolle über ihre personenbezogenen Daten zu gewährleisten \cite{Kuzina.2023}. % 1 
Unternehmen sind also sowohl intrinsisch als auch extrinsisch motiviert, Daten zu schützen. Intern liegt die Motivation darin, das Vertrauen der Kunden zu erhalten und die betriebliche Kontinuität sicherzustellen, während externe Faktoren wie gesetzliche Vorschriften und der Wettbewerbsdruck zusätzliche Anreize bieten, Sicherheitsstandards zu wahren.

% Relevanz Data Leakage Prevention
Ein Hauptziel der Informationssicherheit besteht darin, die Offenlegung von Daten gegenüber Unbefugten zu verhindern. Datenlecks können jedoch nicht immer verhindert werden. Diese Bedrohung kann von böswilligen externen Akteuren ausgehen, die versuchen, an sensible Daten zu gelangen. Gleichzeitig können auch interne Mitarbeiter eine Gefahr darstellen, indem sie Informationen absichtlich oder unbeabsichtigt preisgeben  \cite{Alneyadi.2016}. Bereits im Jahr 2018 zeigten Studien, dass 53\% der befragten Unternehmen in den letzten 12 Monaten von Insider-Angriffen betroffen waren. Dabei sind interne Bedrohungen oft schwerwiegender als externe, da sie in der Regel schwieriger zu erkennen sind \cite{CATechnologies.2018}.
Die Offenlegung von sensiblen Daten kann erheblichen Schaden verursachen. Unternehmen können ihren Wettbewerbsvorteil verlieren, ihr Image beeinträchtigen, Umsatzeinbußen erleiden oder sogar Geldstrafen und Sanktionen erhalten.

Um das Risiko von Datenschutzverletzungen zu minimieren, werden zunehmend Data-Leakage-Prevention (DLP) Lösungen eingesetzt. Gartner prognostiziert, dass bis 2027 etwa 70\% der größeren Unternehmen eine DLP-Lösung einsetzen werden, um die Datensicherheit vor Insider-Risiken und externen Angreifern zu schützen \cite{Chugh.2023}. DLP-Systeme überwachen den Zugriff und Austausch vertraulicher Daten, um unbefugte Offenlegung oder missbräuchliche Nutzung zu erkennen.

% Fortschritt in KI hilft auch Cloud Security
Unternehmen sammeln oft große Datenmengen, ohne genau zu wissen, welche Informationen erfasst werden. Auch die Suche nach und der Abruf von personenbezogenen Daten ist dabei eine große Herausforderung. Dies erschwert den Schutz der Privatsphäre. DLP-Systeme benötigen Informationen darüber, ob bestimmte Daten besonders schützenswert sind oder nicht. Im Zeitalter von Big Data ist es jedoch kaum noch möglich, die enormen Datenmengen manuell zu analysieren. Der Fortschritt im Bereich künstlicher Intelligenz (KI) bietet hierbei einen vielversprechenden Ansatz. KI-basierte Methoden zur automatischen Datenklassifizierung können in DLP-Systemen eingesetzt werden, um sensible Informationen zu erkennen.

% Fokus der Arbeit, Struktur
Aufgrund der neuen Möglichkeiten durch den Einsatz von KI im Bereich Datenschutz liegt der Fokus in dieser Arbeit auf der Anwendung von Methoden der automatischen Datenklassifizierung zur Erkennung sensibler Informationen. Die zentrale Fragestellung dieser Arbeit befasst sich mit der effektiven Erkennung sensibler Daten in großen Datenmengen. Herbei wird zunächst die Bedrohung durch versehentliche Offenlegung von Daten beschrieben und anschließend die Abwehrmaßnahme 'Data Leakage Prevention' vorgestellt. Dabei liegt der Fokus auf der Erkennung von sensiblen Informationen. Es werden verschiedene KI-basierte Methoden und ihre Funktionsweise im Bezug auf Datenklassifizierung vorgestellt. Anschließend wird deren Einsatz in der Cloud Sicherheit diskutiert.