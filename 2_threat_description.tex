% Einführung der Threats
Die Cloud Security Alliance (CSA) veröffentlicht jährlich einen Bericht über die größten Bedrohungen der Cloud Security. Dabei werden über 700 Experten zu verschiedenen Themen in der Cloud Security befragt. Ziel dieses Berichts ist es, auf Bedrohungen, Risiken und Schwachstellen in der Cloud aufmerksam zu machen. Der aktuellste Bericht von 2022 zeigt, dass sich die Verantwortung bezüglich der Sicherheit in der Cloud weg vom Cloud Service Provider und hin zum Cloud-Kunden bewegt \cite{CloudSecurityAlliance.2022}. % S. 6-8
Durch die Verlagerung der Verantwortung und die Komplexität der Cloud steigt das Risiko von Fehlern durch Unwissen. Deshalb ist das achte Sicherheitsproblem des Berichts 'Accidental Cloud Data Disclosure'.

% Accidental Cloud Data Discloure 
Die versehentliche Offenlegung von Cloud Daten ist eine Sicherheitsbedrohung, die auftritt, wenn sensible Informationen unbeabsichtigt öffentlich zugänglich gemacht werden. Dies kann durch menschliches Versagen, Konfigurationsfehler oder unzureichende Sicherheitsmaßnahmen verursacht werden \cite{CloudSecurityAlliance.2022}. % S. 33-34
% Beispiele
So wurde beispielsweise 2023 bekannt, dass bei Toyota Motor die persönlichen Daten von Kunden über mehrere Jahre offengelegt wurden. Der Grund war eine Fehlkonfiguration, wodurch die Datenbank in der Cloud öffentlich zugänglich war \cite{Whittaker.2023}.
Im August 2023 wurden personenbezogene Daten der derzeitigen Beamten des nordirischen Polizeidienstes versehentlich von einem internen Mitarbeiter auf einer Online-Plattform veröffentlicht, der die Datei verwechselt hatte \cite{PSNI.2023}.


% Angriffe
Allein die beiden Beispiele zeigen, wie schnell Fehler passieren und so großer Schaden angerichtet werden kann. Durch die Verlagerung der Sicherheits-Verantwortung auf die Cloud-Kunden steigt das Risiko von menschlichem Versagen. Durch Social Engineering und Phishing-Attacken können Mitarbeiter eines Unternehmens unbeabsichtigt sensible Daten wie Zugangsdaten offenlegen. Wie im Beispiel der Polizei in Nordirland kann es Mitarbeitern auch passieren, Daten unwissentlich zu veröffentlichen. Auch der Verlust von unzureichend geschützten Geräten wie Laptop oder Smartphone kann zu einer Daten Offenlegung führen. Der einfache Zugang zu Cloud-Ressourcen kann außerdem dazu verleiten, neue Ressourcen anzulegen oder Services zu nutzen, ohne sich über die nötigen Möglichkeiten der Absicherung zu informieren, wodurch Daten durch Fehlkonfigurationen offengelegt werden können. Doch nicht nur der Mensch ist ein Risiko, sondern auch das Zielsystem. Durch schwache Passwörter, fehlende Authentifizierung bei sicherheitsrelevanten Systemen und weitere Fehlkonfigurationen können Daten in der Cloud unwissentlich offengelegt werden. Aber auch ungeschlossene Sicherheitslücken in verwendeten Cloud-Services sind ein Sicherheitsrisiko \cite{Trabelsi.2019}\cite{Brindha.2015}.

% Abwehrmaßnahmen
Um die Risiken für eine versehentliche Offenlegung von Daten zu minimieren, gibt es verschiedene Schutzmaßnahmen. Mit einem kontrollierten Identity Access Management (IAM) kann der interne und externe Zugriff auf die Daten geregelt und kontrolliert werden. Mit einer genauen Kontrolle der Zugriffsrechte auf die Cloud-Ressourcen können unbefugte Benutzer daran gehindert werden, auf sensible Daten zuzugreifen. Durch die Einführung von strengen Passwortrichtlinien und dem Einsatz von Passwort-Manager-Software kann das Risiko des unbefugten Zugriffs auf Geräte, Benutzerkonten oder Cloud-Ressourcen minimiert werden. Durch den Einsatz des Prinzips des geringsten Privilegs erhalten Benutzer zudem nur die Berechtigungen, die sie unmittelbar für ihre Aufgaben benötigen. Das minimiert das Risiko von Fehlkonfigurationen oder missbräuchlichem Zugriff. Neben der Kontrolle der Zugriffe sollten auch die möglichen Schwachstellen überwacht werden. Regelmäßige Schwachstellen-Scans helfen dabei, Sicherheitslüclen in der Cloud-Infrastruktur zu identifizieren und zu beheben, bevor diese ausgenutzt werden können. Die Überprüfung und Optimierung von Cloud-Konfigurationen gewährleistet, dass Sicherheitseinstellungen korrekt konfiguriert sind. Zudem ermöglicht eine zentrale Aufstellung und Management aller in der Cloud vorhandenen Assets eine bessere Kontrolle und Überwachung der Daten, Dienste und Einstellungen. Um bekannte Sicherheitslücken zu schließen, sollte eingesetzte Software regelmäßig aktualisiert werden. Um menschliche Fehler zu minimieren, sollten außerdem die Mitarbeiter mit Schulungen für sicherheitsrelevante Themen sensibilisiert werden \cite{Brindha.2015}.

% Überleitung Cloud Data Leakage Prevention
Im Bezug auf diese Bedrohung werden die Begriffe Data Loss (Datenverlust) und Data Leakage (Datenleck) häufig als Synonym verwendet, aber sie haben einige Unterschiede. Datenverlust ist der Verlust von Daten, der nicht wiederherstellbar ist, wie z.B. durch Schäden an Speichermedien, unbeabsichtigtes Löschen oder Hardwarefehler. Datenlecks hingegen beziehen sich auf die unbeabsichtigte oder absichtliche Übertragung von Daten aus einem gesicherten Bereich. Daher können Datenlecks auftreten, wenn unbefugte Personen sensible oder vertrauliche Informationen erhalten \cite{Proofpoint.2021b}. Aus diesem Grund wird in dieser Arbeit der Ausdruck Datenleck oder Data Leakage verwendet, um die unbeabsichtigte Offenlegung von Daten zu beschreiben.


