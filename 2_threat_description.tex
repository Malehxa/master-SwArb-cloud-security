% Einführung der Threats
Die Cloud Security Alliance (CSA) veröffentlicht jährlich einen Bericht über die größten Bedrohungen der Cloud Security, der auf der Befragung von über 700 Experten basiert. Ziel dieses Berichts ist es, auf Bedrohungen, Risiken und Schwachstellen in der Cloud aufmerksam zu machen. Der aktuellste Bericht von 2022 hebt hervor, dass die Verantwortung für die Sicherheit in der Cloud vermehrt vom Cloud Service Provider zum Cloud-Kunden verlagert wird \cite{CloudSecurityAlliance.2022}. % S. 6-8
Diese Verschiebung erhöht das Risiko von Fehlern aufgrund von Unwissenheit. Ein zentrales Sicherheitsproblem ist die unbeabsichtigte Offenlegung von Cloud-Daten ('Accidental Cloud Data Disclosure').

% Accidental Cloud Data Disclosure 
Die versehentliche Offenlegung von Cloud-Daten ist eine Sicherheitsbedrohung, bei der sensible Informationen unbeabsichtigt öffentlich zugänglich gemacht werden. Das kann durch menschliches Versagen, Konfigurationsfehler oder unzureichende Sicherheitsmaßnahmen verursacht werden \cite{CloudSecurityAlliance.2022}. % S. 33-34
% Beispiele
Ein Beispiel ist der Fall von Toyota Motor im Jahr 2023, bei dem persönlichen Daten von Kunden über mehrere Jahre offengelegt wurden. Der Grund war eine Fehlkonfiguration, wodurch die Datenbank in der Cloud öffentlich zugänglich war \cite{Whittaker.2023}.
Ein weiteres Ereignis im August 2023 betraf die nordirische Polizei, als ein interner Mitarbeiter versehentlich persönliche Daten der aktuellen Beamten auf einer Online-Plattform veröffentlichte, indem er die Datei verwechselte \cite{PSNI.2023}.

% Angriffe
Die beiden genannten Beispiele verdeutlichen die potenziell schwerwiegenden Folgen von Fehlern, die schnell zu erheblichem Schaden führen können. Mit der steigenden Verantwortung für die Sicherheit in der Cloud seitens der Kunden erhöht sich das Risiko menschlichen Versagens. Social Engineering und Phishing-Attacken stellen eine Gefahr dar, da Mitarbeiter unbeabsichtigt sensible Daten wie Zugangsdaten offenlegen können. Wie im Fall der nordirischen Polizei gezeigt, besteht auch das Risiko, dass Mitarbeiter Daten unwissentlich veröffentlichen. Zu einer ungewollten Offenlegung von Daten kann es auch kommen, wenn Geräte wie Laptops oder Smartphones, die nicht ausreichend geschützt sind, verloren gehen. Der einfache Zugang zu Cloud-Ressourcen kann dazu führen, dass neue Ressourcen oder Dienste ohne ausreichende Sicherheitsüberlegungen genutzt werden, wodurch Daten aufgrund von Fehlkonfigurationen offengelegt werden könnten. Nicht nur menschliches Versagen, sondern auch Schwächen im Zielsystem stellen Risiken dar. Schwache Passwörter, mangelnde Authentifizierung bei sicherheitsrelevanten Systemen und andere Konfigurationsfehler können bewirken, dass Daten in der Cloud unbeabsichtigt offengelegt werden. Ebenso stellen ungeschlossene Sicherheitslücken in genutzten Cloud-Services eine Bedrohung dar \cite{Trabelsi.2019}\cite{Brindha.2015}.

% Abwehrmaßnahmen
Um das Risiko einer versehentlichen Datenpreisgabe zu minimieren, können verschiedene Schutzmaßnahmen ergriffen werden. Ein kontrolliertes Identity Access Management (IAM) ermöglicht die Regulierung und Kontrolle des internen und externen Datenzugriffs. Die Einführung strenger Passwortrichtlinien und die Nutzung von Passwort-Manager-Software reduzieren das Risiko unbefugten Zugriffs auf Geräte, Benutzerkonten oder Cloud-Ressourcen. Das Prinzip des geringsten Privilegs gewährleistet, dass Benutzer nur die notwendigen Berechtigungen für ihre Aufgaben erhalten, was das Risiko von Fehlkonfigurationen oder missbräuchlichem Zugriff minimiert. Neben der Kontrolle der Zugriffe ist auch die Überwachung möglicher Schwachstellen entscheidend. Regelmäßige Schwachstellen-Scans helfen, Sicherheitslücken in der Cloud-Infrastruktur zu identifizieren und zu beheben, bevor sie ausgenutzt werden können. Die Überprüfung und Optimierung von Cloud-Konfigurationen gewährleistet korrekte Sicherheitseinstellungen. Eine zentrale Verwaltung aller in der Cloud vorhandenen Assets ermöglicht eine bessere Kontrolle und Überwachung von Daten, Diensten und Einstellungen. Regelmäßige Softwareaktualisierungen sind wichtig, um bekannte Sicherheitslücken zu schließen. Mitarbeiter sollten zudem durch Schulungen für sicherheitsrelevante Themen sensibilisiert werden, um menschliche Fehler zu minimieren \cite{Brindha.2015}.

% Überleitung Cloud Data Leakage Prevention
Im Kontext dieser Bedrohung werden die Begriffe Data Loss (Datenverlust) und Data Leakage (Datenleck) häufig als Synonyme verwendet, weisen jedoch einige Unterschiede auf. Datenverlust bezieht sich auf den unwiederbringlichen Verlust von Daten, beispielsweise durch Schäden an Speichermedien, unbeabsichtigtes Löschen oder Hardwarefehler. Im Gegensatz dazu bezeichnet Datenleck die unbeabsichtigte oder absichtliche Übertragung von Daten aus einem gesicherten Bereich. Datenlecks können auftreten, wenn unbefugte Personen sensible oder vertrauliche Informationen erhalten \cite{Proofpoint.2021b}. Daher wird in dieser Arbeit der Begriff Datenleck oder Data Leakage verwendet, um die unbeabsichtigte Offenlegung von Daten zu beschreiben.

Die zunehmende Komplexität der Cloud-Infrastrukturen und die steigende Verantwortung der Kunden für die Sicherheit haben das Risiko der versehentlichen Offenlegung sensibler Daten erheblich erhöht. In Anbetracht dieser Herausforderungen und um das Risiko von Datenlecks zu minimieren, werden DLP-Systeme zunehmend als entscheidende Sicherheitsmaßnahme eingesetzt.


