Diese Arbeit gibt einen Überblick über verschiedene Ansätze im Bereich der Cloud Data Leakage Prevention unter Verwendung von automatischen Datenklassifizierungsmethoden. Zu Beginn wurde die Problematik der versehentlichen Offenlegung von Cloud-Daten analysiert. Anschließend wurden verschiedene Ansätze zur Prävention von Datenlecks in der Cloud vorgestellt, wobei der Schwerpunkt auf der Nutzung von Cloud-DLP-Systemen lag. Dabei hat sich herausgestellt, dass die Erkennung sensibler Daten von verschiedenen Datenmerkmalen beeinflusst wird, darunter Kontext, Kategorien, Vertraulichkeitsstufen, Struktur und Zustände. Im Fokus stand die automatische Datenklassifizierung, die als entscheidendes Element für den Schutz sensibler Informationen betrachtet wurde. Hierbei wurden verschiedene Methoden wie die Klassifizierung mit manueller Definition betrachtet, die regelbasierte Methoden und Data Fingerprinting umfasst. Zudem wurde die Anwendung von KI-basierten Methoden und maschinellem Lernen in der automatischen Datenklassifizierung untersucht, wobei Algorithmen wie k-NN, Boosting, Clusteranalyse und ein kontextbasierter Ansatz für die Erkennung strukturierter sensibler Daten betrachtet wurden. Mit der reinen Klassifizierung von sensiblen Daten sind diese jedoch noch nicht geschützt. Deshalb wurden im letzten Kapitel noch einige Schutzmaßnahmen betrachtet, die auf klassifizierten Daten basieren.

Besonders die manuelle Datenklassifizierung und auch die Klassifizierung mit manueller Definition werden zunehmend risikobehaftet. Bei der Anwendung manueller Sicherheitsrichtlinien besteht die Gefahr, dass nicht alle relevanten und sensiblen Daten einbezogen werden, und die Richtlinien möglicherweise nicht kontinuierlich aktualisiert werden. Des Weiteren erfordert der manuelle Aufwand, dass geschultes Sicherheitspersonal diese Aufgaben übernimmt. Allerdings verfügen diese Mitarbeiter nicht über umfassendes Wissen über sämtliche Daten innerhalb des Unternehmens.
Die Literaturrecherche verdeutlicht, dass die automatische Datenklassifizierung bereits erfolgreich angewendet wurde und aktuell erforscht wird, um die Klassifizierung zu verbessern und effizienter zu gestalten. Insbesondere der verstärkte Einsatz von Cloud-Technologien stellt die Data Leakage Prevention vor neue Herausforderungen. Die Verteilung von Daten auf verschiedene Medien, der schnelle Zugriff und die großen Datenmengen sind nur einige davon. Trotz der herausragenden Ergebnisse des Einsatzes von Methoden aus dem KI-Bereich stehen Unternehmen damit vor neue Herausforderungen. Die Entwicklung eines maschinellen Lernmodells oder eines neuronalen Netzes zur Erkennung sensibler Daten hängt stark von der Verfügbarkeit realer Datensätze mit vertraulichen Daten ab. Allerdings sind solche Datensätze nicht immer für die Öffentlichkeit zugänglich und könnten zudem nicht alle notwendigen Eigenschaften für ein Unternehmen enthalten. In datenschutzrelevanten Bereichen werden daher häufig synthetische Datensätze verwendet. Diese Datensätze können sowohl für das Training des Modells als auch zur Erweiterung eines realen Datensatzes verwendet werden \cite{Kuzina.2023}.

Die fortlaufende Entwicklung im Bereich der Cloud Data Leakage Prevention und automatischen Datenklassifizierung bietet vielversprechende Perspektiven. Die erörterten Ansätze bieten einen umfassenden Überblick über Möglichkeiten zur Verhinderung von Datenlecks in der Cloud, wobei die automatische Datenklassifizierung als zentrales Element für den Schutz sensibler Informationen herausgestellt wurde. Es wird deutlich, dass die Herausforderungen im Zusammenhang mit der verstärkten Nutzung von Cloud-Technologien und der Datenverteilung auf verschiedene Medien weiterhin im Fokus stehen. Die Anwendung von KI und maschinellem Lernen zeigt vielversprechende Ergebnisse.
Für die Zukunft der Forschung und Implementierung in diesem Bereich könnten weiterführende Studien den Fokus auf die Optimierung von automatischen Datenklassifizierungsmethoden legen, um die Effizienz und Genauigkeit zu steigern. Zudem könnten innovative Ansätze zur Integration von Schutzmaßnahmen auf Basis der klassifizierten Daten weiter erforscht werden, um einen umfassenden Schutz sensibler Informationen in der Cloud zu gewährleisten.