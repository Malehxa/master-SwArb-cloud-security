Diese Arbeit gibt einen Überblick über verschiedene Ansätze im Bereich der Cloud Data Leakage Prevention mit Methoden der automatischen Datenklassifizierung. Zunächst wird die Problematik der versehentlichen Offenlegung von Cloud-Daten analysiert. Anschließend werden verschiedene Ansätze zur Prävention von Datenlecks in der Cloud vorgestellt mit Schwerpunkt auf der Nutzung von Cloud-DLP-Systemen. Dabei hat sich herausgestellt, dass die Erkennung sensibler Daten beeinflusst wird von den verschiedenen möglichen Eigenschaften von Daten, wie Kontext, Kategorien, Vertraulichkeitsstufen, Struktur und Zuständen. Im Fokus steht die automatische Datenklassifizierung, die als entscheidendes Element für den Schutz sensibler Informationen betrachtet wird. Hierbei werden verschiedene Methoden betrachtet, darunter die Klassifizierung mit manueller Definition, die regelbasierte Methoden und Data Fingerprinting umfasst. Zudem wird die Anwendung von maschinellem Lernen in der automatischen Datenklassifizierung untersucht, wobei Algorithmen wie k-NN, Boosting, Clusteranalyse und ein kontextbasierter Ansatz für die Erkennung strukturierter sensibler Daten betrachtet wurden. Mit der reinen Klassifizierung von sensiblen Daten sind diese jedoch noch nicht geschützt. Deshalb wurden im letzten Kapitel noch einige Schutzmaßnahmen betrachtet, die auf klassifizierten Daten basieren.

Besonders die manuelle Datenklassifizierung und auch die Klassifizierung von manuellem Zutun werden immer risikoreicher. Bei Verwendung von manuellen Sicherheitsrichtlinien, besteht die Gefahr, dass nicht alle wichtigen und sensiblen Daten einbezogen werden und die Richtlinien nicht stetig aktualisiert werden. Außerdem bedeutet manueller Aufwand, dass dies durch geschultes Sicherheitspersonal durchgeführt werden muss. Doch diese Mitarbeiter sind auch nicht allwissend über die gesamten Daten innerhalb des Unternehmens.
Anhand der Literatur-Recherche hat sich gezeigt, dass sich die automatische Datenklassifizierung bewährt hat und vor allem daran geforscht wird, inwiefern die Klassifizierung verbessert und effizienter gemacht werden kann. Insbesondere der verstärkte Einsatz von Cloud-Technologien stellt die Data Leakage Prevention vor neue Herausforderungen. Die Verteilung von Daten auf verschiedene Medien, der schnelle Zugriff und die großen Datenmengen sind nur einige davon. Trotz der herausragenden Ergebnisse des Einsatzes von Methoden des maschinellen Lernens, stellt es Unternehmen auch vor neue Herausforderungen. Die Entwicklung eines maschinellen Lernmodells zur Erkennung sensibler Daten hängt stark von der Verfügbarkeit realer Datensätze mit vertraulichen Daten ab. Allerdings sind diese Datensätze nicht immer für die Öffentlichkeit zugänglich und enthalten möglicherweise auch nicht alle für ein Unternehmen nötigen Eigenschaften. In Datenschutz-relevanten Bereichen werden deshalb oft synthetische Datensätze verwendet. Diese Datensätze können sowohl zum Trainieren des Modells, als auch zur Erweiterung eines realen Datensatzes verwendet werden \cite{Kuzina.2023}.

Die vorgestellten Ansätze bieten vielversprechende Möglichkeiten für die Verbesserung der Cloud-Sicherheit. Technologische Fortschritte bieten der Cloud-Sicherheit neue Möglichkeiten, jedoch auch neue Herausforderungen.