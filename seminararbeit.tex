\documentclass[conference]{IEEEtran}
\IEEEoverridecommandlockouts
% The preceding line is only needed to identify funding in the first footnote. If that is unneeded, please comment it out.
\usepackage{cite}
\usepackage{amsmath,amssymb,amsfonts}
\usepackage{algorithmic}
\usepackage{graphicx}
\usepackage{textcomp}
\usepackage{xcolor}
\usepackage{balance}
\usepackage{multirow}
\usepackage{bigstrut}
\usepackage{graphicx}
\usepackage{tabularx}
\usepackage{tabulary}

\usepackage[utf8]{inputenc}
\usepackage[T1]{fontenc} % Trennen von Wörtern mit Umlauten
\usepackage{german} % Damit z.B. "Literatur" statt "References" da steht

%\renewcommand{\arraystretch}{1.5}

\def\BibTeX{{\rm B\kern-.05em{\sc i\kern-.025em b}\kern-.08em
    T\kern-.1667em\lower.7ex\hbox{E}\kern-.125emX}}

\begin{document}

\title{Cloud Data Leakage Prevention mit Methoden der automatischen Datenklassifizierung}

\author{
    \IEEEauthorblockN{Anna Hamberger}
    \IEEEauthorblockA{\textit{Fakultät für Informatik} \\
        \textit{Technische Hochschule Rosenheim}\\
        Rosenheim, Germany \\
        anna.hamberger@stud.th-rosenheim.de}
}

\maketitle

% abstract & keywords
\begin{abstract}
    Die Cloud wird aufgrund ihrer Flexibilität, Skalierbarkeit und kostengünstigen Ressourcenverwaltung immer beliebter. Mit dem Einsatz von Cloud-Diensten steigt jedoch auch die Komplexität. Zudem verschiebt sich die Verantwortung zum Nutzer. Diese Faktoren können in Unternehmen zu unbeabsichtigten Datenlecks führen. Der Einsatz von Systemen zur Verhinderung von Datenlecks in der Cloud wird immer wichtiger, um Datenschutzverletzungen in der Cloud zu verhindern. Diese Systeme sind darauf ausgelegt, sensible Daten zu erkennen und zu klassifizieren sowie Verstöße gegen Sicherheitsrichtlinien zu verhindern und zu melden. Ein wesentlicher Aspekt dabei ist die automatische Klassifizierung von sensiblen Daten. Durch die zunehmenden Datenmengen im Kontext von Big Data sind effiziente Technologien und Methoden erforderlich. Der Fortschritt im Bereich künstlicher Intelligenz (KI) bietet hierbei einen vielversprechenden Ansatz. KI-basierte Methoden zur automatischen Datenklassifizierung können lernen, sensible Informationen zu erkennen und zu klassifizieren. Die Verwendung von Methoden wie k-NN, Boosting, Clusteranalyse und Sprachverarbeitungsmodelle haben dabei gute Ergebnisse erzielt. Cloud-Data-Leakage-Prevention-Systeme können auf Basis der klassifizierten Daten verschiedene Sicherheitsmaßnahmen ergreifen und so den Schutz vor unbeabsichtigten Datenlecks verbessern.


\end{abstract}

\begin{IEEEkeywords}
    component, formatting, style, styling, insert
\end{IEEEkeywords}

% chapters
\section{Einführung}
% Einleitung warum Cloud immer beliebter wird
Im Jahr 2022 gaben bereits 84\% der befragten 552 Unternehmen in Deutschland an, Cloud-Dienste zu verwenden \cite{KPMG.2022}. Cloud Computing hat sich während der digitalen Transformation zu einem wichtigen Bestandteil der Informationsverarbeitung entwickelt. Die Beliebtheit von Cloud-Diensten nimmt stetig zu, da sie die effiziente Speicherung großer Datenmengen, den schnellen Zugang zu Ressourcen und den nahtlosen Datenaustausch ermöglichen. Mit der Verbreitung digitaler Technologien in der Gesellschaft und in Unternehmen werden zunehmend mehr Daten geteilt. Unternehmen nutzen die Vorteile von Cloud-Diensten, um diese umfangreichen Datenmengen zu sammeln und zu verarbeiten. Die Möglichkeit, Daten in Echtzeit zu teilen, verbessert Geschäftsprozesse und erleichtert die unternehmensinterne Zusammenarbeit \cite{Surianarayanan.2023b}. %18-27

% warum Datensicherheit
Da Informationen das wertvollste Gut eines Unternehmens darstellen, ist ihr Schutz von größter Bedeutung. Beim Sammeln von Daten ist ein Unternehmen zudem verpflichtet, diese vor Diebstahl, Verlust und Missbrauch zu schützen. Zahlreiche Datenschutzgesetze und -vorschriften, wie die EU-Datenschutz-Grundverordnung (DSGVO), wurden eingeführt, um sensible Daten wie personenbezogene Informationen zu sichern. Ziel dieser Vorschriften ist es, klare Richtlinien für das Sammeln von Daten festzulegen und Einzelpersonen eine vergleichsweise hohe Kontrolle über ihre personenbezogenen Daten zu gewährleisten \cite{Kuzina.2023}. % 1 
Unternehmen sind also sowohl intrinsisch als auch extrinsisch motiviert, Daten zu schützen. Intern liegt die Motivation darin, das Vertrauen der Kunden zu erhalten und die betriebliche Kontinuität sicherzustellen, während externe Faktoren wie gesetzliche Vorschriften und der Wettbewerbsdruck zusätzliche Anreize bieten, Sicherheitsstandards zu wahren.

% Relevanz Data Leakage Prevention
Ein Hauptziel der Informationssicherheit besteht darin, die Offenlegung von Daten gegenüber Unbefugten zu verhindern. Datenlecks können jedoch nicht immer verhindert werden. Diese Bedrohung kann von böswilligen externen Akteuren ausgehen, die versuchen, an sensible Daten zu gelangen. Gleichzeitig können auch interne Mitarbeiter eine Gefahr darstellen, indem sie Informationen absichtlich oder unbeabsichtigt preisgeben  \cite{Alneyadi.2016}. Bereits im Jahr 2018 zeigten Studien, dass 53\% der befragten Unternehmen in den letzten 12 Monaten von Insider-Angriffen betroffen waren. Dabei sind interne Bedrohungen oft schwerwiegender als externe, da sie in der Regel schwieriger zu erkennen sind \cite{CATechnologies.2018}.
Die Offenlegung von sensiblen Daten kann erheblichen Schaden verursachen. Unternehmen können ihren Wettbewerbsvorteil verlieren, ihr Image beeinträchtigen, Umsatzeinbußen erleiden oder sogar Geldstrafen und Sanktionen erhalten.

Um das Risiko von Datenschutzverletzungen zu minimieren, werden zunehmend Data-Leakage-Prevention (DLP) Lösungen eingesetzt. Gartner prognostiziert, dass bis 2027 etwa 70\% der größeren Unternehmen eine DLP-Lösung einsetzen werden, um die Datensicherheit vor Insider-Risiken und externen Angreifern zu schützen \cite{Chugh.2023}. DLP-Systeme überwachen den Zugriff und Austausch vertraulicher Daten, um unbefugte Offenlegung oder missbräuchliche Nutzung zu erkennen.

% Fortschritt in KI hilft auch Cloud Security
Unternehmen sammeln oft große Datenmengen, ohne genau zu wissen, welche Informationen erfasst werden. Auch die Suche nach und der Abruf von personenbezogenen Daten ist dabei eine große Herausforderung. Dies erschwert den Schutz der Privatsphäre. DLP-Systeme benötigen Informationen darüber, ob bestimmte Daten besonders schützenswert sind oder nicht. Im Zeitalter von Big Data ist es jedoch kaum noch möglich, die enormen Datenmengen manuell zu analysieren. Der Fortschritt im Bereich künstlicher Intelligenz (KI) bietet hierbei einen vielversprechenden Ansatz. KI-basierte Methoden zur automatischen Datenklassifizierung können in DLP-Systemen eingesetzt werden, um sensible Informationen zu erkennen.

% Fokus der Arbeit, Struktur
Aufgrund der neuen Möglichkeiten durch den Einsatz von KI im Bereich Datenschutz liegt der Fokus in dieser Arbeit auf der Anwendung von Methoden der automatischen Datenklassifizierung zur Erkennung sensibler Informationen. Die zentrale Fragestellung dieser Arbeit befasst sich mit der effektiven Erkennung sensibler Daten in großen Datenmengen. Herbei wird zunächst die Bedrohung durch versehentliche Offenlegung von Daten beschrieben und anschließend die Abwehrmaßnahme 'Data Leakage Prevention' vorgestellt. Dabei liegt der Fokus auf der Erkennung von sensiblen Informationen. Es werden verschiedene KI-basierte Methoden und ihre Funktionsweise im Bezug auf Datenklassifizierung vorgestellt. Anschließend wird deren Einsatz in der Cloud Sicherheit diskutiert.

\section{Accidental Cloud Data Disclosure} \label{threat-kapitel}
% Einführung der Threats
Die Cloud Security Alliance (CSA) veröffentlicht jährlich einen Bericht über die größten Bedrohungen der Cloud Security. Dabei werden über 700 Experten zu verschiedenen Themen in der Cloud Security befragt. Ziel dieses Berichts ist es, auf Bedrohungen, Risiken und Schwachstellen in der Cloud aufmerksam zu machen. Der aktuellste Bericht von 2022 zeigt, dass sich die Verantwortung bezüglich der Sicherheit in der Cloud weg vom Cloud Service Provider und hin zum Cloud-Kunden bewegt \cite{CloudSecurityAlliance.2022}. % S. 6-8
Durch die Verlagerung der Verantwortung und die Komplexität der Cloud steigt das Risiko von Fehlern durch Unwissen. Deshalb ist das achte Sicherheitsproblem des Berichts 'Accidental Cloud Data Disclosure'.

% Accidental Cloud Data Discloure 
Die versehentliche Offenlegung von Cloud Daten ist eine Sicherheitsbedrohung, die auftritt, wenn sensible Informationen unbeabsichtigt öffentlich zugänglich gemacht werden. Dies kann durch menschliches Versagen, Konfigurationsfehler oder unzureichende Sicherheitsmaßnahmen verursacht werden \cite{CloudSecurityAlliance.2022}. % S. 33-34
% Beispiele
So wurde beispielsweise 2023 bekannt, dass bei Toyota Motor die persönlichen Daten von Kunden über mehrere Jahre offengelegt wurden. Der Grund war eine Fehlkonfiguration, wodurch die Datenbank in der Cloud öffentlich zugänglich war \cite{Whittaker.2023}.
Im August 2023 wurden personenbezogene Daten der derzeitigen Beamten des nordirischen Polizeidienstes versehentlich von einem internen Mitarbeiter auf einer Online-Plattform veröffentlicht, der die Datei verwechselt hatte \cite{PSNI.2023}.


% Angriffe
Allein die beiden Beispiele zeigen, wie schnell Fehler passieren und so großer Schaden angerichtet werden kann. Durch die Verlagerung der Sicherheits-Verantwortung auf die Cloud-Kunden steigt das Risiko von menschlichem Versagen. Durch Social Engineering und Phishing-Attacken können Mitarbeiter eines Unternehmens unbeabsichtigt sensible Daten wie Zugangsdaten offenlegen. Wie im Beispiel der Polizei in Nordirland kann es Mitarbeitern auch passieren, Daten unwissentlich zu veröffentlichen. Auch der Verlust von unzureichend geschützten Geräten wie Laptop oder Smartphone kann zu einer Daten Offenlegung führen. Der einfache Zugang zu Cloud-Ressourcen kann außerdem dazu verleiten, neue Ressourcen anzulegen oder Services zu nutzen, ohne sich über die nötigen Möglichkeiten der Absicherung zu informieren, wodurch Daten durch Fehlkonfigurationen offengelegt werden können. Doch nicht nur der Mensch ist ein Risiko, sondern auch das Zielsystem. Durch schwache Passwörter, fehlende Authentifizierung bei sicherheitsrelevanten Systemen und weitere Fehlkonfigurationen können Daten in der Cloud unwissentlich offengelegt werden. Aber auch ungeschlossene Sicherheitslücken in verwendeten Cloud-Services sind ein Sicherheitsrisiko \cite{Trabelsi.2019}\cite{Brindha.2015}.

% Abwehrmaßnahmen
Um die Risiken für eine versehentliche Offenlegung von Daten zu minimieren, gibt es verschiedene Schutzmaßnahmen. Mit einem kontrollierten Identity Access Management (IAM) kann der interne und externe Zugriff auf die Daten geregelt und kontrolliert werden. Mit einer genauen Kontrolle der Zugriffsrechte auf die Cloud-Ressourcen können unbefugte Benutzer daran gehindert werden, auf sensible Daten zuzugreifen. Durch die Einführung von strengen Passwortrichtlinien und dem Einsatz von Passwort-Manager-Software kann das Risiko des unbefugten Zugriffs auf Geräte, Benutzerkonten oder Cloud-Ressourcen minimiert werden. Durch den Einsatz des Prinzips des geringsten Privilegs erhalten Benutzer zudem nur die Berechtigungen, die sie unmittelbar für ihre Aufgaben benötigen. Das minimiert das Risiko von Fehlkonfigurationen oder missbräuchlichem Zugriff. Neben der Kontrolle der Zugriffe sollten auch die möglichen Schwachstellen überwacht werden. Regelmäßige Schwachstellen-Scans helfen dabei, Sicherheitslüclen in der Cloud-Infrastruktur zu identifizieren und zu beheben, bevor diese ausgenutzt werden können. Die Überprüfung und Optimierung von Cloud-Konfigurationen gewährleistet, dass Sicherheitseinstellungen korrekt konfiguriert sind. Zudem ermöglicht eine zentrale Aufstellung und Management aller in der Cloud vorhandenen Assets eine bessere Kontrolle und Überwachung der Daten, Dienste und Einstellungen. Um bekannte Sicherheitslücken zu schließen, sollte eingesetzte Software regelmäßig aktualisiert werden. Um menschliche Fehler zu minimieren, sollten außerdem die Mitarbeiter mit Schulungen für sicherheitsrelevante Themen sensibilisiert werden \cite{Brindha.2015}.

% Überleitung Cloud Data Leakage Prevention
Im Bezug auf diese Bedrohung werden die Begriffe Data Loss (Datenverlust) und Data Leakage (Datenleck) häufig als Synonym verwendet, aber sie haben einige Unterschiede. Datenverlust ist der Verlust von Daten, der nicht wiederherstellbar ist, wie z.B. durch Schäden an Speichermedien, unbeabsichtigtes Löschen oder Hardwarefehler. Datenlecks hingegen beziehen sich auf die unbeabsichtigte oder absichtliche Übertragung von Daten aus einem gesicherten Bereich. Daher können Datenlecks auftreten, wenn unbefugte Personen sensible oder vertrauliche Informationen erhalten \cite{Proofpoint.2021b}. Aus diesem Grund wird in dieser Arbeit der Ausdruck Datenleck oder Data Leakage verwendet, um die unbeabsichtigte Offenlegung von Daten zu beschreiben.




\section{Cloud Data Leakage Prevention System}
% Aufnahme Data Leakage Prevention in ISO
Im vorherigen Kapitel \ref{threat-kapitel} über die Bedrohung durch versehentliche Datenoffenlegung wurden deutlich, wie schnell ein Datenleck in Unternehmen auftreten kann. Die Bedeutung effektiver Maßnahmen zur Vermeidung von Datenlecks hat sich aufgrund der wachsenden Datenmengen und des damit verbundenen Risikos einer Datenschutzverletzung erhöht. Dieser Bedarf wurde 2022 erkannt, als in der neuesten Version der Norm ISO 27001:2022 die Data Leakage Prevention eingeführt wurde. Die internationale Norm ISO 27001 definiert die Bedingungen für die Einrichtung, Umsetzung und kontinuierliche Verbesserung eines dokumentierten Informationssicherheits-Managementsystems. Die Norm gibt außerdem Vorschriften für die Beurteilung und Behandlung von Informationssicherheitsrisiken, die an die spezifischen Bedürfnisse jedes Unternehmens angepasst werden müssen \cite{Monev.2023}.
% DLP Abkürzung vorher schon eingeführt?
Ein Datenleck kann auf verschiedene Weise auftreten. Trotz der Tatsache, dass es nicht immer möglich ist, das Auftreten vollständig zu verhindern, können Maßnahmen ergriffen werden, um die Wahrscheinlichkeit eines Auftretens zu verringern. Diese Maßnahmen werden als Data Leakage Prevention (DLP) bezeichnet \cite{Monev.2023}. Dabei handelt es sich um eine Reihe von Technologien, Produkten und Methoden, die dazu dienen, zu verhindern, dass vertrauliche Informationen ein Unternehmen verlassen. In den letzten Jahrzehnten wurden verschiedene Sicherheitssysteme wie Firewalls, Intrusion-Detection-Systeme (Einbrucherkennung) und virtuelle private Netzwerke (VPN) eingeführt, um das Risiko von Datenlecks zu reduzieren. Wenn die zu schützenden Daten klar definiert, strukturiert und konstant sind, erfüllen diese Systeme ihren Zweck. Jedoch sind sie unzuverlässig für Daten, die sich ändern oder unstrukturiert sind. Durch einfache Regeln kann beispielsweise eine Firewall den Zugriff auf ein sensibles Datenobjekt verhindern. Die Firewall erkennt jedoch nicht, wenn das Datenobjekt über einen E-Mail-Anhang gesendet wird. DLP-Systeme hingegen sind darauf spezialisiert, vertrauliche Daten zu identifizieren, zu überwachen und zu schützen und unerwünschte Datenbewegungen zu verhindern.\cite{Alneyadi.2016}.

\subsection{Cloud Data Leakage Prevention System}
Ein DLP-System umfasst eine Reihe von Regeln und Richtlinien, die Daten nach ihrem Typ klassifizieren, um sicherzustellen, dass sie nicht böswillig oder versehentlich weitergegeben werden. Das System überwacht Endbenutzeraktivitäten, den Datenfluss sowie die über das Netzwerk gesendeten Daten. Wenn verdächtige Aktivitäten erkannt werden, wird eine Systemwarnung ausgelöst. DLP-Lösungen identifizieren sensible Inhalte mithilfe von Datenklassifizierung-Label, Techniken zur Inspektion von Inhalten und Kontextanalysen. Sie überwachen die Datenaktivität und kontrollieren sie anhand vordefinierter DLP-Richtlinien. Die Richtlinien definieren, ob die Verwendung bestimmter Inhalte oder Daten in bestimmten Situationen erlaubt sind \cite{Chugh.2023}.

Gartner klassifiziert DLP-Lösungen in drei Kategorien. Eine Enterprise-DLP-Lösung ist ein zentrales System, das darauf ausgelegt ist, komplexe Anforderungen und Strukturen großer Unternehmen zu bewältigen. Sie verfügt über fortschrittliche Technologien zur Identifikation, Klassifizierung und Markierung sensibler Daten und ist in der Lage, verschiedene Datenquellen zu integrieren. So kann diese Lösung den gesamten Lebenszyklus von Daten in einem Unternehmen abdecken. DLP-Richtlinien werden dabei an zentraler Stelle verwaltet und durchgesetzt.
Dagegen werden integrierte DLP-Lösungen direkt in einen Dienst, wie bspw. ein E-Mail-Gateway, integriert und verfügen deshalb nur über begrenzte Richtlinienfunktionen. Das Management von mehreren integrierten DLP-Systemen ist ein manueller Aufwand, jedoch werden diese Systeme im jeweiligen Dienst speziell an die Anforderungen angepasst und können Inhaltsüberprüfungen besser durchführen.
Cloud-native DLP-Lösungen sind die dritte Kategorie, zu der sowohl SaaS-Lösungen als auch Cloud-Anbieter mit integrierten DLP-Funktionen gehören. Sie sind speziell für den Einsatz in Cloud-Umgebungen entwickelt und darauf ausgerichtet, sensible Daten in Cloud-Diensten zu schützen. Diese Lösungen verfügen über Mechanismen zur automatischen Erkennung von sensiblen Daten, die in Cloud-Anwendungen und -Speicherplätzen gespeichert sind. Dies umfasst die Identifikation von Daten in Form von Dokumenten, E-Mails, Datenbanken und anderen Formaten \cite{Chugh.2023}.
Im weiteren Verlauf der Arbeit wird der Begriff DLP-System für alle drei Kategorien verwendet.

% Aufbau / Funktionen
% Bild von NIST oder DLP Funktionen einfügen?
Das Cybersecurity Framework des National Institute of Standards and Technology (NIST CSF) bietet freiwillige Standards und Best Practices, die Unternehmen dabei helfen, Cybersecurity-Risiken zu managen und zu reduzieren. Es gibt Unternehmen eine Struktur, um ihre aktuelle Cybersicherheitssituation zu bewerten, verbesserungsbedürftige Bereiche zu identifizieren, Maßnahmen zu priorisieren, Fortschritte zu bewerten und mit den Stakeholdern zu kommunizieren. Die CSF besteht aus fünf Kernfunktionen: Identifizieren, Schützen, Erkennen, Reagieren und Wiederherstellen.
DLP-Systeme konzentrieren sich hauptsächlich auf die Identifizierung, die Erkennung und den Schutz und ergänzen diese Funktionen durch den Bereich der Überwachung. Die spezifischen Funktionen eines DLP-Systems können je nach Hersteller variieren \cite{NIST.2014}.

% TODO: Literatur-Recherche  + Ergebnisse als Tabelle darstellen?
Die Literatur-Recherche ergab die folgende Auswahl an Best-Practises, die in DLP-Systemen eingesetzt werden sollten. Um sensible Daten schützen zu können, müssen diese zuerst identifiziert werden. Die Aufgabe besteht darin, ein Dateninventar zu erstellen, die Daten nach ihrer Sensibilität zu klassifizieren und sie entsprechend zu kennzeichnen. Zum Schutz der sensiblen Daten sollten Maßnahmen ergriffen werden, die den Zugriff auf die Daten einschränken. Das bedeutet, dass Richtlinien wie minimale Zugriffsrechte, starke Authentifizierungsmethoden und strenge Zugriffskontrolllisten eingeführt werden sollten. Außerdem sollten Daten sowohl im Ruhezustand als auch während der Übertragung verschlüsselt werden. So wird sichergestellt, dass die Daten selbst dann, wenn sie abgefangen werden, für unbefugte Benutzer unlesbar bleiben. Zusätzlich sollte ein DLP-System die Datenströme innerhalb und nach außen überwachen, um potenzielle Datenschutzverletzungen oder Richtlinienverstöße in Echtzeit erkennen zu können. Dies ermöglicht eine schnelle Reaktion auf potenzielle Probleme und begrenzt den daraus resultierenden Schaden \cite{Hussain.2022}\cite{HerreraMontano.2022}\cite{Shishodia.2022}. % es gibt noch weitere Möglichkeiten?

% Herausforderung Daten Klassifizierung
Die Funktionen eines DLP-Systems basieren alle darauf, dass sensible Daten erkannt und in irgendeiner Art markiert sind. Der erste Schritt bei DLP-Systemen ist daher die Identifizierung sensibler Daten. Es gibt verschiedene Strategien und Methoden zur Klassifizierung dieser Daten, die durch den Einsatz von KI weiter verbessert wurden.

\subsection{Erkennung von sensiblen Daten}
% Warum Datenklassifizierung? 
% TODO: SaaS schon eingeführt?
Unternehmen setzen immer mehr auf SaaS-Produkte, anstatt sie als Produkt zu kaufen \cite{Gartner.2023}. In ihrem Tagesgeschäft verlassen sich Unternehmen oft auf mehrere Softwareprodukte, um verschiedene Anforderungen zu erfüllen. Das hat zur Folge, dass die Daten des Unternehmens über verschiedene Apps und Cloud-Plattformen verstreut sind. Die Herausforderung besteht darin, den Überblick zu behalten und zu wissen, wo sich die sensiblen Daten befinden. Das Sammeln und Identifizieren von Daten in DLP-Systemen stellt aufgrund von Verschlüsselung, verborgenen Kanälen, nicht unterstützten Datenformaten und großer Mengen an Daten eine große Herausforderung dar \cite{Hauer.2015}.

Die Methoden zur Erkennung und Klassifizierung von sensiblen Daten unterscheiden sich je nach Art und Format der Daten, sowie deren Zustand. Außerdem gibt es die Möglichkeit, Daten manuell oder automatisiert zu klassifizieren.

\subsubsection{Eigenschaften von Daten}
% Art -> welche Kategorien
Sensible Daten durchdringen fast jeden Aspekt unseres persönlichen und beruflichen Lebens. Das Spektrum dieser sensiblen Informationen reicht von persönlichen Daten über Finanzinformationen und Geschäftsgeheimnisse bis hin zu biometrischen Merkmalen und umfasst eine Vielzahl von Kategorien. Guo, Liu et al. \cite{Guo.2021} kategorisiert beispielsweise Daten in die vier Bereiche:
\begin{itemize}
    \item Persönliche Informationen (z.B. Name, Geburtsdatum oder Gesundheitsinformationen)
    \item Informationen zur Netzwerkidentität (z.B. IP-Adresse, MAC-Adresse oder E-Mail)
    \item Vertrauliche und Anmeldeinformationen (z.B. Login-Passwort-Kombinationen, API-Token oder digitale Zertifikate)
    \item Finanzinformationen (z.B. Bankkontodaten, Kreditkarteninformationen oder Verbrauchsdaten)
\end{itemize}
Die Kategorisierung und der Detailgrad können je nach Unternehmen variieren.

Neben den Kategorien muss auch der Kontext beachtet werden, in dem eine Information verwendet wird. Denn der Kontext hat direkten Einfluss auf die Sensibilität. Pogiatzis und Samakovitis \cite{Pogiatzis.2020} leiten vier verschiedene Kontextklassen ab, die auf der Bedeutung, der Interaktion, der Priorität und Präferenz basieren, die mit jeder Information verbunden sind.

\begin{itemize}
    \item Der semantische Kontext wird auf Grundlage der semantischen Bedeutung eines Begriffs gebildet. Die semantische Bedeutung einer Sequenz wirkt sich zum Beispiel auf ihre Sensibilität aus.
    \item Im Kontext der Akteure wird die Sensibilität von Daten abhängig von den Akteuren, die an der Informationsübermittlung beteiligt sind, betrachtet. Die Sensibilität wird durch die Beziehung zwischen den beteiligten Akteuren bestimmt. Zum Beispiel ist der Austausch von Gesundheitsinformationen zwischen Patient und Arzt nicht sensibel, außerhalb dieser Gruppe von Akteuren jedoch schon.
    \item Der zeitliche Kontext bezieht sich auf die Priorität der Informationen, die die Bedeutung des Begriffs beeinflussen. Eine Zeichenfolge, die als Passwort eingegeben wird, gilt bspw. als vertraulicher, als wenn sie als Benutzername eingegeben wird.
    \item Der Selbstkontext wird durch die persönlichen Präferenzen des Nutzers in Bezug auf seine Privatsphäre bestimmt. Zum Beispiel kann eine Person ihre ethnische Herkunft als vertrauliche Information betrachten, eine andere nicht.
\end{itemize}
Auch hier können verschiedene kontextuelle Kategorien unterschieden oder definieren werden. Sie sind zudem nicht immer klar trennbar und schließen sich nicht gegenseitig aus. Ein oder mehrere Kontexte können sich gleichzeitig unterschiedlich auf die Sensibilität auswirken. Manche Daten können jedoch auch unabhängig vom Kontext vertraulich sein, wie z.B. Passwörter oder Kreditkartennummern.

% Daten Klassen
Die Einteilung von Daten in verschiedene Geheimhaltungsklassen ist ein häufig verwendetes Verfahren in militärischen und behördlichen Anwendungen. Militärische Anwendungen verwenden dabei Begriffe wie \glqq eingeschränkt\grqq , \glqq vertraulich\grqq, \glqq geheim\grqq und \glqq streng geheim\grqq \cite{Landwehr.1984}. So ist es möglich, sensible Daten noch präziser in Vertraulichkeitsstufen zu unterteilen.

% Format -> structured, unstructured
Außerdem wird die Klassifizierung auch von der Struktur der Daten beeinflusst. Mehr als 80\% der Daten im Internet bestehen aus unstrukturierten Daten \cite{Allahyari.2017}. Unstrukturierte Daten beziehen sich in der Regel auf Informationen, die nicht in einer relationalen Datenbank gespeichert sind. Folglich gibt es kein vordefiniertes Datenmodell und die Struktur ist unregelmäßig oder unvollständig. Selbst Datenformate wie CSV, JSON oder XML, die einige organisatorische Eigenschaften haben, verfügen in der Regel nicht über ein klar definiertes Datenmodell. Im Vergleich zu strukturierten Daten ist es schwieriger, unstrukturierte Daten abzurufen, zu analysieren und zu speichern. Während Menschen unstrukturierte Daten leicht verarbeiten können, haben Maschinen oft Schwierigkeiten damit \cite{Guo.2021}.

Die Herausforderung bei der Datenklassifizierung besteht daher darin, die Kategorien, den Kontext, die Vertraulichkeitsstufen und die Struktur der Daten zu berücksichtigen.

\subsubsection{Datenzustand}
Im Bereich der Informationssicherheit werden Daten je nach ihrem Zustand unterschiedlich betrachtet. Die verschiedenen Datenzustände helfen dabei, die geeigneten Sicherheitsmaßnahmen zu bestimmen. Im Rahmen der Data Leakage Prevention können sich Daten in einem der drei Zustände befinden: Daten im Ruhezustand, Daten in Bewegung und Daten in Verwendung \cite{Shabtai.2012b}.

Daten im Ruhezustand sind Daten, die auf einem physischen oder digitalen Speichermedium gespeichert sind, wie bspw. einer Datenbank, auf einer Festplatte, im Cloud-Speicher oder einem externen Datenspeicher. Ruhende Daten sind in der Regel inaktiv und werden nicht aktiv gelesen oder gerade übertragen. Sicherheitsmaßnahmen wie Verschlüsselung, Authentifizierung und Zugriffskontrollregeln werden von DLP-Systemen zu Erkennungs- und Überwachungszwecken eingesetzt \cite{Shabtai.2012b}\cite{Shishodia.2022}.

Daten in Bewegung hingegen beziehen sich auf den Zustand von Daten, wenn sie gerade aktiv über Netzwerke oder andere Kommunikationskanäle übertragen werden oder sich im Speicher eines Computers befinden und zum Lesen, Aktualisieren und Verarbeiten bereit sind. Beispiele für Daten in dieser Kategorie sind Daten, die über das Internet, soziale Medien, E-Mail oder FTP/SSH übertragen werden. Daten, die über das Netzwerk übertragen werden, sollten mit Verschlüsselungsmethoden wie HTTPS, SSL oder TLS geschützt werden. DLP-Systeme nutzen Erkennungs- und Überwachungsfunktionen, um den Datenfluss durch das Netzwerk zu identifizieren und zu überprüfen \cite{Shabtai.2012b}\cite{Shishodia.2022}.

Daten in Verwendung sind Daten, die eine Person oder ein System aktiv verarbeiten, aktualisieren, anhängen oder löschen. Dabei befinden sich die Daten auf der Workstation, dem Laptop, dem USB-Stick oder der externen Festplatte des Endnutzers sowie auf Netzlaufwerken oder Druckern. Diese Art von Daten ist besonders anfällig für Sicherheitsbedrohungen, da sie aktiv manipuliert werden. Diese Daten sind über verschiedene Endpunkte hinweg sichtbar, wenn auf sie zugegriffen wird. Um diese Daten zu schützen, wird über das DLP-System eine starke Benutzerauthentifizierung sowie ein Identitäts- und Profilmanagement eingeführt. Außerdem kann je nach DLP-Technologie ein Endpunkt-Agent auf dem Gerät des Endnutzers installiert werden, um die Datennutzung und -übertragung zu überwachen \cite{Shabtai.2012b}\cite{Shishodia.2022}.

Die Unterscheidung zwischen drei Zuständen von Daten - in Ruhe, in Bewegung und in Verwendung - ist entscheidend für die Identifizierung sensibler Informationen und die Klassifizierung von Daten. Mit der Berücksichtigung der Datenzustände können sensible Daten differenziert betrachtet und klassifiziert werden.


\subsubsection{Manuelle Datenklassifizierung}
% Warum manuell
Viele Studien in der Literatur haben Klassifizierungsmethoden verwendet, um die Datensicherheit in der Cloud zu gewährleisten. Die vorgeschlagenen Lösungen lassen sich in zwei Klassen einteilen: die manuelle Klassifizierung, die vom Nutzer festgelegt wird, und die automatische Klassifizierung, bei der ein Algorithmus zum Einsatz kommt.

Die manuelle Datenklassifizierung ist trotz Fortschritte bei automatisierten Technologien eine gängige Methode. In einigen Fällen, wie bei der Klassifizierung von Informationen wie geistigem Eigentum oder Geschäftsgeheimnissen, bleibt die manuelle Klassifizierung erforderlich. Menschen sind in der Lage, leicht die Kategorien, Kontexte, Strukturen und Zustände der Daten ganzheitlich zu berücksichtigen. Außerdem kann die manuelle Klassifizierung für kleinere Unternehmen kostengünstig sein \cite{Divadari.2023}\cite{Alsuwaie.2021}.

% warum schwierig (aufwändig, fehlerbehaftet, Big Data)
Die manuelle Klassifizierung von Daten ist jedoch sehr anfällig für menschliche Fehler und Inkonsistenzen. Die Subjektivität der Menschen kann zu inkonsistenten Klassifizierungen führen, was die Genauigkeit und Zuverlässigkeit der Sicherheitsmaßnahmen beeinträchtigen kann. Zudem kann eine unzureichende Genauigkeit zu unvollständigen oder falschen Klassifizierungen führen. Im Normalfall werden Daten bei ihrer Erstellung klassifiziert, doch sie können sich im Laufe der Zeit ändern, wodurch die ursprüngliche Klassifizierung veraltet und nicht mehr richtig sein kann. Die manuelle Einordnung von großen Datenmengen kann sehr zeitaufwändig und arbeitsintensiv sein und ab einer bestimmten Größe nicht mehr manuell verarbeitet werden. Manuelle Prozesse können die Anpassungsfähigkeit und schnelle Reaktion auf sich ändernde Geschäftsanforderungen reduzieren \cite{Venhorst.2019}.

Aufgrund der wachsenden Datenmengen und der zunehmenden Komplexität der Informationssicherheitsanforderungen erscheint die automatisierte Datenklassifizierung häufig als effizientere Lösung, die eine genauere und konsistentere Identifizierung sensibler Daten ermöglicht.

\subsubsection{Automatische Datenklassifizierung}
% warum automatisiert
Der Prozess, bei dem maschinelle Algorithmen und Technologien verwendet werden, um Daten automatisch zu identifizieren, zu kategorisieren und entsprechend ihres Sensitivität zu klassifizieren, wird als automatische Datenklassifizierung bezeichnet. Diese Methode verwendet maschinelles Lernen, Mustererkennung oder künstliche Intelligenz, um Daten zu analysieren und automatisch geeignete Klassifizierungen zuzuweisen.
Die aktuellen Techniken in der Data Leakage Prevention können allgemein in zwei Kategorien eingeteilt werden: inhaltsbasierte Analyse und kontextbasierte Analyse. Methoden, die auf dem Inhalt basieren, untersuchen den Inhalt von Daten anhand von Merkmalen sensibler Informationen wie regulären Ausdrücken und Datenfingerabdrücken. Inhaltsbasierte Methoden verwenden vorhersehbare Muster wie z.B. IP-Adressen oder E-Mail-Adressen, um sensible Daten zu erkennen. Kontextbasierte Techniken identifizieren vertrauliche Daten anhand von Merkmalen im Zusammenhanf mit den überwachten Daten. Der kontextbasierte Ansatz ist damit effektiver für vertrauliche Daten ohne vorhersehbare Muster \cite{Guo.2021}\cite{Gugelmann.2015}\cite{Kuzina.2023}. Um sensible Informationen umfassender und genauer zu extrahieren, sollten daher verschiedene Methoden angewendet werden. In der Tabelle \ref{t:methoden} sind die verschiedenen Methoden nach ihrer Kategorie aufgelistet, die in der Literatur verwendet wurden.

%Tabelle mit content-based vs. context-based
\begin{table}[htbp]
    \normalsize
    \caption{Methoden der automatischen Datenklassifizierung. Quelle: eigene Darstellung.}
    \label{t:methoden}
    \begin{center}
        \begin{tabulary}{15cm}{|L|L|}
            \hline
            \textbf{Kategorie}                           & \textbf{Methode} \bigstrut  \\
            \hline
            \hline
            \multirow{9}{4em}{basierend auf dem Inhalt}  & rule-based \bigstrut[t]     \\
            & ML Classifier               \\
            & Vector based                \\
            & Frequency method            \\
            & Fingerprint                 \\
            & Neuronal Network            \\
            & Statistic Analysis          \\
            & Text Clustering             \\
            & ML K-NN                     \\
            \hline
            \multirow{5}{4em}{basierend auf dem Kontext} & Deep Learning  \bigstrut[t] \\
            & Graph-based                 \\
            & ML semantic analysis        \\
            & BiLSTM                      \\
            & Cassed                      \\
            \hline
        \end{tabulary}
    \end{center}
\end{table}

\paragraph{content-based}
- rule-based/regular-matching/Dictionary
- ML classifier
- Vector based
- Frequency method
- fingerprint
- neuronal network
- statistical analysis
- text clustering
- ML K-NN

\paragraph{context-based}
- deep learning
- graph based -> CBDLPA
- ML semantic analysis
- BERT-BiLSTM-Attention Model (Guo)
- BERT: Cassed


% Data preprocessing


\subsection{Anwendung in der Cloud Security}
% wofür ist das dann gut

\subsubsection{Labeling für andere Maßnahmen}
% verschlüsselung
% Access Rights
% Versenden

\subsubsection{Anwendungsbeispiele}
% BrowserFlow
% DocGuard?

\section{Ausblick}
% Stärken und Schwächen

% Verbesserung im Kampf

% references ieeetr
\balance
\bibliographystyle{IEEEtran}
\bibliography{ref}

\end{document}
