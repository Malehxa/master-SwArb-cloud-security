\documentclass[conference]{IEEEtran}
\IEEEoverridecommandlockouts
% The preceding line is only needed to identify funding in the first footnote. If that is unneeded, please comment it out.
\usepackage{cite}
\usepackage{amsmath,amssymb,amsfonts}
\usepackage{algorithmic}
\usepackage{graphicx}
\usepackage{textcomp}
\usepackage{xcolor}
\usepackage{balance}
\usepackage{multirow}
\usepackage{bigstrut}
\usepackage{graphicx}
\usepackage{tabularx}
\usepackage{tabulary}
\usepackage{caption}
\usepackage{subcaption}
\usepackage{url}

\usepackage[utf8]{inputenc}
\usepackage[T1]{fontenc} % Trennen von Wörtern mit Umlauten
\usepackage{german} % Damit z.B. "Literatur" statt "References" da steht


%\renewcommand{\arraystretch}{1.5}

\def\BibTeX{{\rm B\kern-.05em{\sc i\kern-.025em b}\kern-.08em
    T\kern-.1667em\lower.7ex\hbox{E}\kern-.125emX}}

\def\figdir{figures}

\begin{document}

\title{Cloud Data Leakage Prevention mit Methoden der automatischen Datenklassifizierung}

\author{
    \IEEEauthorblockN{Anna Hamberger}
    \IEEEauthorblockA{\textit{Fakultät für Informatik} \\
        \textit{Technische Hochschule Rosenheim}\\
        Rosenheim, Germany \\
        anna.hamberger@stud.th-rosenheim.de}
}

\maketitle

% abstract & keywords
\begin{abstract}
    Die Nutzung von Cloud-Diensten wird immer beliebter, da sie große Datenmengen effizient speichern, schnellen Zugriff auf Ressourcen bieten und einen nahtlosen Datenaustausch ermöglichen. Allerdings steigt damit auch das Risiko, dass Daten in der Cloud ungewollt offengelegt werden. Das Problem der Datenverluste ist zu einer kritischen Herausforderung geworden, die die Entwicklung umfassender Systeme in diesem Bereich erforderlich macht. Ein System zur Verhinderung von Datenlecks (Data Leakage Prevention, DLP) konzentriert sich darauf, sensible Daten zu identifizieren und klassifizieren und Verstöße gegen Sicherheitsrichtlinien zu verhindern und zu melden. Ein wesentlicher Aspekt dieses Systems ist die automatische Klassifizierung von sensiblen Daten. Es gibt zwar manuelle Methoden, aber maschinelle Lernverfahren haben sich als äußerst effektiv erwiesen. Cloud-DLP-Systeme können auf der Grundlage der klassifizierten Daten verschiedene Sicherheitsmaßnahmen ergreifen und so den Schutz vor Datenlecks verbessern.


\end{abstract}

\begin{IEEEkeywords}
    machine learning, data classification, data leakage prevention, cloud security
\end{IEEEkeywords}

% chapters
\section{Einführung}
% Einleitung warum Cloud immer beliebter wird
Im Jahr 2022 gaben bereits 84\% der befragten 552 Unternehmen in Deutschland an, dass sie Cloud-Dienste in ihrem Unternehmen einsetzen \cite{KPMG.2022}. Cloud Computing hat sich in der Zeit der digitalen Transformation zu einem wichtigen Bestandteil der Informationsverarbeitung entwickelt. Die Nutzung von Cloud-Diensten wird immer beliebter, da sie die Möglichkeit zur effizienten Speicherung großer Datenmengen, schnellen Zugang zu Ressourcen und nahtlosen Datenaustausch bietet. Durch den Verbreitung von digitaler Technologie in der Gesellschaft und in Unternehmen werden immer mehr Daten geteilt und gesammelt. Um diese großen Datenmengen sammeln und verarbeiten zu können, nutzen Unternehmen die Vorteile von Cloud-Diensten. Die Möglichkeit, Daten in Echtzeit zu teilen, verbessert Geschäftsprozesse und erleichtert die Zusammenarbeit im Unternehmen \cite{Surianarayanan.2023b}. %18-27

% warum Datensicherheit
Da Informationen das wertvollste Gut eines Unternehmens sind, ist ihr Schutz von größter Bedeutung. Beim Sammeln von Daten ist ein Unternehmen zudem verpflichtet, sie vor Diebstahl, Verlust und Missbrauch zu schützen. Es gibt zahlreiche Datenschutzgesetze und -vorschriften, wie die EU-Datenschutz-Grundverordnung (DSGVO), um sensible Daten wie personenbezogene Daten zu schützen. Ziel dieser Vorschriften ist es, strengen Regeln für das Sammeln von Daten vorzugeben und der Einzelperson eine vergleichsweise hohe Kontrolle über ihre personenbezogenen Daten zu geben \cite{Kuzina.2023}. % 1
Unabhängig des Speicherorts besteht also das Risiko, dass die Datensicherheit verletzt wird.

% Relevanz Data Leakage Prevention
Eines der Hauptziele der Informationssicherheit ist die Verhinderung der Offenlegung von Daten gegenüber Unbefugten. Datenlecks können jedoch aufgrund der Notwendigkeit, auf Informationen zuzugreifen, diese zu teilen und zu nutzen, nicht immer verhindert werden. Diese Bedrohung kann von böswilligen Außenstehenden ausgehen, die versuchen, sensible Daten zu erhalten. Umgekehrt können auch interne Mitarbeiter eine Gefahr darstellen, wenn sie beabsichtigt oder unbeabsichtigt Informationen preisgeben \cite{Alneyadi.2016}. Bereits im Jahr 2018 haben Studien gezeigt, dass 53\% der befragten Unternehmen Insider-Angriffe in den letzten 12 Monaten bestätigten. Dabei sind Bedrohungen von innen häufig schwerwiegender als von außen, da sie meist schwieriger zu erkennen sind \cite{CATechnologies.2018}.
Die Offenlegung von sensiblen Daten kann erheblichen Schaden verursachen. Unternehmen können ihren Wettbewerbsvorteil verlieren, ihr Image beeinträchtigen, Umsatzeinbußen erleiden oder sogar Geldstrafen und Sanktionen erhalten.

Um das Risiko von Datenschutzverletzungen zu minimieren, werden immer häufiger Data-Leakage-Prevention (DLP) Lösungen eingesetzt. Gartner prognostiziert, dass bis 2027 etwa 70\% der größeren Unternehmen eine DLP-Lösung einsetzen werden, um die Datensicherheit vor Insider-Risiken und externen Angreifern zu schützen \cite{Chugh.2023}. DLP-Systeme überwachen den Zugriff und Austausch vertraulicher Daten, um unbefugte Offenlegung oder missbräuchliche Nutzung zu erkennen.

% Fortschritt in KI hilft auch Cloud Security
Unternehmen sammeln häufig große Datenmengen, ohne zu wissen, was erfasst wird oder wie sie nach personenbezogenen Daten suchen oder diese abrufen können. Das erschwert den Schutz der Privatsphäre. DLP-Systeme benötigen die Information, ob bestimmte Daten besonders schützenswert sind oder nicht. Im Zeitalter von Big Data ist es jedoch kaum noch möglich, die enormen Datenmengen manuell zu analysieren. Der Fortschritt im Bereich künstliche Intelligenz (KI) bietet hierbei einen vielversprechenden Ansatz. KI-basierte Methoden zur automatischen Datenklassifizierung können in DLP-Systemen eingesetzt werden, um sensible Informationen zu erkennen.

% Fokus der Arbeit, Struktur
Aufgrund der neuen Möglichkeiten mit dem Einsatz von KI im Bereich Datenschutz liegt der Fokus in dieser Arbeit auf der Anwendung von Methoden der automatischen Datenklassifizierung zur Erkennung sensibler Informationen, um den Schutz sensibler Daten zu gewährleisten. Diese Arbeit beschäftigt sich mit der Frage, wie sensible Daten in große Datenmengen am besten erkannt werden können. Dabei wird zunächst die Bedrohung durch versehentliche Offenlegung von Daten beschrieben und anschließend die Abwehrmaßnahme 'Data Leakage Prevention' vorgestellt. Dabei liegt der Fokus auf der Erkennung von sensiblen Informationen. Es werden verschiedene KI-basierte Methoden und ihre Funktionsweise im Bezug auf Datenklassifizierung vorgestellt. Anschließend wird deren Einsatz in der Cloud Sicherheit diskutiert.

\section{Versehentliche Offenlegung von Daten in der Cloud} \label{threat-kapitel}
% Einführung der Threats
Die Cloud Security Alliance (CSA) veröffentlicht jährlich einen Bericht über die größten Bedrohungen der Cloud Security, der auf der Befragung von über 700 Experten basiert. Ziel dieses Berichts ist es, auf Bedrohungen, Risiken und Schwachstellen in der Cloud aufmerksam zu machen. Der aktuellste Bericht von 2022 hebt hervor, dass die Verantwortung für die Sicherheit in der Cloud vermehrt vom Cloud Service Provider zum Cloud-Kunden verlagert wird \cite{CloudSecurityAlliance.2022}. % S. 6-8
Diese Verschiebung erhöht das Risiko von Fehlern aufgrund von Unwissenheit. Ein zentrales Sicherheitsproblem ist die unbeabsichtigte Offenlegung von Cloud-Daten ('Accidental Cloud Data Disclosure').

% Accidental Cloud Data Disclosure 
Die versehentliche Offenlegung von Cloud-Daten ist eine Sicherheitsbedrohung, bei der sensible Informationen unbeabsichtigt öffentlich zugänglich gemacht werden. Das kann durch menschliches Versagen, Konfigurationsfehler oder unzureichende Sicherheitsmaßnahmen verursacht werden \cite{CloudSecurityAlliance.2022}. % S. 33-34
% Beispiele
Ein Beispiel ist der Fall von Toyota Motor im Jahr 2023, bei dem persönlichen Daten von Kunden über mehrere Jahre offengelegt wurden. Der Grund war eine Fehlkonfiguration, wodurch die Datenbank in der Cloud öffentlich zugänglich war \cite{Whittaker.2023}.
Ein weiteres Ereignis im August 2023 betraf die nordirische Polizei, als ein interner Mitarbeiter versehentlich persönliche Daten der aktuellen Beamten auf einer Online-Plattform veröffentlichte, indem er die Datei verwechselte \cite{PSNI.2023}.

% Angriffe
Die beiden genannten Beispiele verdeutlichen die potenziell schwerwiegenden Folgen von Fehlern, die schnell zu erheblichem Schaden führen können. Mit der steigenden Verantwortung für die Sicherheit in der Cloud seitens der Kunden erhöht sich das Risiko menschlichen Versagens. Social Engineering und Phishing-Attacken stellen eine Gefahr dar, da Mitarbeiter unbeabsichtigt sensible Daten wie Zugangsdaten offenlegen können. Wie im Fall der nordirischen Polizei gezeigt, besteht auch das Risiko, dass Mitarbeiter Daten unwissentlich veröffentlichen. Zu einer ungewollten Offenlegung von Daten kann es auch kommen, wenn Geräte wie Laptops oder Smartphones, die nicht ausreichend geschützt sind, verloren gehen. Der einfache Zugang zu Cloud-Ressourcen kann dazu führen, dass neue Ressourcen oder Dienste ohne ausreichende Sicherheitsüberlegungen genutzt werden, wodurch Daten aufgrund von Fehlkonfigurationen offengelegt werden könnten. Nicht nur menschliches Versagen, sondern auch Schwächen im Zielsystem stellen Risiken dar. Schwache Passwörter, mangelnde Authentifizierung bei sicherheitsrelevanten Systemen und andere Konfigurationsfehler können bewirken, dass Daten in der Cloud unbeabsichtigt offengelegt werden. Ebenso stellen ungeschlossene Sicherheitslücken in genutzten Cloud-Services eine Bedrohung dar \cite{Trabelsi.2019}\cite{Brindha.2015}.

% Abwehrmaßnahmen
Um das Risiko einer versehentlichen Datenpreisgabe zu minimieren, können verschiedene Schutzmaßnahmen ergriffen werden. Ein kontrolliertes Identity Access Management (IAM) ermöglicht die Regulierung und Kontrolle des internen und externen Datenzugriffs. Die Einführung strenger Passwortrichtlinien und die Nutzung von Passwort-Manager-Software reduzieren das Risiko unbefugten Zugriffs auf Geräte, Benutzerkonten oder Cloud-Ressourcen. Das Prinzip des geringsten Privilegs gewährleistet, dass Benutzer nur die notwendigen Berechtigungen für ihre Aufgaben erhalten, was das Risiko von Fehlkonfigurationen oder missbräuchlichem Zugriff minimiert. Neben der Kontrolle der Zugriffe ist auch die Überwachung möglicher Schwachstellen entscheidend. Regelmäßige Schwachstellen-Scans helfen, Sicherheitslücken in der Cloud-Infrastruktur zu identifizieren und zu beheben, bevor sie ausgenutzt werden können. Die Überprüfung und Optimierung von Cloud-Konfigurationen gewährleistet korrekte Sicherheitseinstellungen. Eine zentrale Verwaltung aller in der Cloud vorhandenen Assets ermöglicht eine bessere Kontrolle und Überwachung von Daten, Diensten und Einstellungen. Regelmäßige Softwareaktualisierungen sind wichtig, um bekannte Sicherheitslücken zu schließen. Mitarbeiter sollten zudem durch Schulungen für sicherheitsrelevante Themen sensibilisiert werden, um menschliche Fehler zu minimieren \cite{Brindha.2015}.

% Überleitung Cloud Data Leakage Prevention
Im Kontext dieser Bedrohung werden die Begriffe Data Loss (Datenverlust) und Data Leakage (Datenleck) häufig als Synonyme verwendet, weisen jedoch einige Unterschiede auf. Datenverlust bezieht sich auf den unwiederbringlichen Verlust von Daten, beispielsweise durch Schäden an Speichermedien, unbeabsichtigtes Löschen oder Hardwarefehler. Im Gegensatz dazu bezeichnet Datenleck die unbeabsichtigte oder absichtliche Übertragung von Daten aus einem gesicherten Bereich. Datenlecks können auftreten, wenn unbefugte Personen sensible oder vertrauliche Informationen erhalten \cite{Proofpoint.2021b}. Daher wird in dieser Arbeit der Begriff Datenleck oder Data Leakage verwendet, um die unbeabsichtigte Offenlegung von Daten zu beschreiben.

Die zunehmende Komplexität der Cloud-Infrastrukturen und die steigende Verantwortung der Kunden für die Sicherheit haben das Risiko der versehentlichen Offenlegung sensibler Daten erheblich erhöht. In Anbetracht dieser Herausforderungen und um das Risiko von Datenlecks zu minimieren, werden DLP-Systeme zunehmend als entscheidende Sicherheitsmaßnahme eingesetzt.




\section{Cloud Data Leakage Prevention} \label{dlp-kapitel}
% Aufnahme Data Leakage Prevention in ISO
Im vorherigen Kapitel \ref{threat-kapitel} über die Bedrohung durch versehentliche Datenoffenlegung wurden deutlich, wie schnell ein Datenleck in Unternehmen auftreten kann. Die Bedeutung effektiver Maßnahmen zur Vermeidung von Datenlecks hat sich aufgrund der wachsenden Datenmengen und des damit verbundenen Risikos einer Datenschutzverletzung erhöht. Dieser Bedarf wurde 2022 erkannt, als in der neuesten Version der Norm ISO 27001:2022 die Data Leakage Prevention eingeführt wurde. Die internationale Norm ISO 27001 definiert die Bedingungen für die Einrichtung, Umsetzung und kontinuierliche Verbesserung eines dokumentierten Informationssicherheits-Managementsystems. Die Norm gibt außerdem Vorschriften für die Beurteilung und Behandlung von Informationssicherheitsrisiken, die an die spezifischen Bedürfnisse jedes Unternehmens angepasst werden müssen \cite{Monev.2023}.
% DLP Abkürzung vorher schon eingeführt?
Ein Datenleck kann auf verschiedene Weise auftreten. Trotz der Tatsache, dass es nicht immer möglich ist, das Auftreten vollständig zu verhindern, können Maßnahmen ergriffen werden, um die Wahrscheinlichkeit eines Auftretens zu verringern. Diese Maßnahmen werden als Data Leakage Prevention (DLP) bezeichnet \cite{Monev.2023}. Dabei handelt es sich um eine Reihe von Technologien, Produkten und Methoden, die dazu dienen, zu verhindern, dass vertrauliche Informationen ein Unternehmen verlassen. In den letzten Jahrzehnten wurden verschiedene Sicherheitssysteme wie Firewalls, Intrusion-Detection-Systeme (Einbrucherkennung) und virtuelle private Netzwerke (VPN) eingeführt, um das Risiko von Datenlecks zu reduzieren. Wenn die zu schützenden Daten klar definiert, strukturiert und konstant sind, erfüllen diese Systeme ihren Zweck. Jedoch sind sie unzuverlässig für Daten, die sich ändern oder unstrukturiert sind. Durch einfache Regeln kann beispielsweise eine Firewall den Zugriff auf ein sensibles Datenobjekt verhindern. Die Firewall erkennt jedoch nicht, wenn das Datenobjekt über einen E-Mail-Anhang gesendet wird. DLP-Systeme hingegen sind darauf spezialisiert, vertrauliche Daten zu identifizieren, zu überwachen und zu schützen und unerwünschte Datenbewegungen zu verhindern.\cite{Alneyadi.2016}.

\subsection{Cloud Data Leakage Prevention System}
Ein DLP-System umfasst eine Reihe von Regeln und Richtlinien, die Daten nach ihrem Typ klassifizieren, um sicherzustellen, dass sie nicht böswillig oder versehentlich weitergegeben werden. Das System überwacht Endbenutzeraktivitäten, den Datenfluss sowie die über das Netzwerk gesendeten Daten. Wenn verdächtige Aktivitäten erkannt werden, wird eine Systemwarnung ausgelöst. DLP-Lösungen identifizieren sensible Inhalte mithilfe von Datenklassifizierung-Label, Techniken zur Inspektion von Inhalten und Kontextanalysen. Sie überwachen die Datenaktivität und kontrollieren sie anhand vordefinierter DLP-Richtlinien. Die Richtlinien definieren, ob die Verwendung bestimmter Inhalte oder Daten in bestimmten Situationen erlaubt sind \cite{Chugh.2023}.

Gartner klassifiziert DLP-Lösungen in drei Kategorien. Eine Enterprise-DLP-Lösung ist ein zentrales System, das darauf ausgelegt ist, komplexe Anforderungen und Strukturen großer Unternehmen zu bewältigen. Sie verfügt über fortschrittliche Technologien zur Identifikation, Klassifizierung und Markierung sensibler Daten und ist in der Lage, verschiedene Datenquellen zu integrieren. So kann diese Lösung den gesamten Lebenszyklus von Daten in einem Unternehmen abdecken. DLP-Richtlinien werden dabei an zentraler Stelle verwaltet und durchgesetzt.
Dagegen werden integrierte DLP-Lösungen direkt in einen Dienst, wie bspw. ein E-Mail-Gateway, integriert und verfügen deshalb nur über begrenzte Richtlinienfunktionen. Das Management von mehreren integrierten DLP-Systemen ist ein manueller Aufwand, jedoch werden diese Systeme im jeweiligen Dienst speziell an die Anforderungen angepasst und können Inhaltsüberprüfungen besser durchführen.
Cloud-native DLP-Lösungen sind die dritte Kategorie, zu der sowohl SaaS-Lösungen als auch Cloud-Anbieter mit integrierten DLP-Funktionen gehören. Sie sind speziell für den Einsatz in Cloud-Umgebungen entwickelt und darauf ausgerichtet, sensible Daten in Cloud-Diensten zu schützen. Diese Lösungen verfügen über Mechanismen zur automatischen Erkennung von sensiblen Daten, die in Cloud-Anwendungen und -Speicherplätzen gespeichert sind. Dies umfasst die Identifikation von Daten in Form von Dokumenten, E-Mails, Datenbanken und anderen Formaten \cite{Chugh.2023}.
Im weiteren Verlauf der Arbeit wird der Begriff DLP-System für alle drei Kategorien verwendet.

% Aufbau / Funktionen
% Bild von NIST oder DLP Funktionen einfügen?
Das Cybersecurity Framework des National Institute of Standards and Technology (NIST CSF) bietet freiwillige Standards und Best Practices, die Unternehmen dabei helfen, Cybersecurity-Risiken zu managen und zu reduzieren. Es gibt Unternehmen eine Struktur, um ihre aktuelle Cybersicherheitssituation zu bewerten, verbesserungsbedürftige Bereiche zu identifizieren, Maßnahmen zu priorisieren, Fortschritte zu bewerten und mit den Stakeholdern zu kommunizieren. Die CSF besteht aus fünf Kernfunktionen: Identifizieren, Schützen, Erkennen, Reagieren und Wiederherstellen.
DLP-Systeme konzentrieren sich hauptsächlich auf die Identifizierung, die Erkennung und den Schutz und ergänzen diese Funktionen durch den Bereich der Überwachung. Die spezifischen Funktionen eines DLP-Systems können je nach Hersteller variieren \cite{NIST.2014}.

% TODO: Literatur-Recherche  + Ergebnisse als Tabelle darstellen?
Die Literatur-Recherche ergab die folgende Auswahl an Best-Practises, die in DLP-Systemen eingesetzt werden sollten. Um sensible Daten schützen zu können, müssen diese zuerst identifiziert werden. Die Aufgabe besteht darin, ein Dateninventar zu erstellen, die Daten nach ihrer Sensibilität zu klassifizieren und sie entsprechend zu kennzeichnen. Zum Schutz der sensiblen Daten sollten Maßnahmen ergriffen werden, die den Zugriff auf die Daten einschränken. Das bedeutet, dass Richtlinien wie minimale Zugriffsrechte, starke Authentifizierungsmethoden und strenge Zugriffskontrolllisten eingeführt werden sollten. Außerdem sollten Daten sowohl im Ruhezustand als auch während der Übertragung verschlüsselt werden. So wird sichergestellt, dass die Daten selbst dann, wenn sie abgefangen werden, für unbefugte Benutzer unlesbar bleiben. Zusätzlich sollte ein DLP-System die Datenströme innerhalb und nach außen überwachen, um potenzielle Datenschutzverletzungen oder Richtlinienverstöße in Echtzeit erkennen zu können. Dies ermöglicht eine schnelle Reaktion auf potenzielle Probleme und begrenzt den daraus resultierenden Schaden \cite{Hussain.2022}\cite{HerreraMontano.2022}\cite{Shishodia.2022}. % es gibt noch weitere Möglichkeiten?

% Herausforderung Daten Klassifizierung
Die Funktionen eines DLP-Systems basieren alle darauf, dass sensible Daten erkannt und in irgendeiner Art markiert sind. Der erste Schritt bei DLP-Systemen ist daher die Identifizierung sensibler Daten. Es gibt verschiedene Strategien und Methoden zur Klassifizierung dieser Daten, die durch den Einsatz von KI weiter verbessert wurden.

\subsection{Erkennung von sensiblen Daten}
% Warum Datenklassifizierung? 
% TODO: SaaS schon eingeführt?
Unternehmen setzen immer mehr auf SaaS-Produkte, anstatt sie als Produkt zu kaufen \cite{Gartner.2023}. In ihrem Tagesgeschäft verlassen sich Unternehmen oft auf mehrere Softwareprodukte, um verschiedene Anforderungen zu erfüllen. Das hat zur Folge, dass die Daten des Unternehmens über verschiedene Apps und Cloud-Plattformen verstreut sind. Die Herausforderung besteht darin, den Überblick zu behalten und zu wissen, wo sich die sensiblen Daten befinden. Das Sammeln und Identifizieren von Daten in DLP-Systemen stellt aufgrund von Verschlüsselung, verborgenen Kanälen, nicht unterstützten Datenformaten und großer Mengen an Daten eine große Herausforderung dar \cite{Hauer.2015}.

Die Methoden zur Erkennung und Klassifizierung von sensiblen Daten unterscheiden sich je nach Art und Format der Daten, sowie deren Zustand. Außerdem gibt es die Möglichkeit, Daten manuell oder automatisiert zu klassifizieren.

\subsubsection{Eigenschaften von Daten}
% Art -> welche Kategorien
Sensible Daten durchdringen fast jeden Aspekt unseres persönlichen und beruflichen Lebens. Das Spektrum dieser sensiblen Informationen reicht von persönlichen Daten über Finanzinformationen und Geschäftsgeheimnisse bis hin zu biometrischen Merkmalen und umfasst eine Vielzahl von Kategorien. Guo, Liu et al. \cite{Guo.2021} kategorisiert beispielsweise Daten in die vier Bereiche:
\begin{itemize}
    \item Persönliche Informationen (z.B. Name, Geburtsdatum oder Gesundheitsinformationen)
    \item Informationen zur Netzwerkidentität (z.B. IP-Adresse, MAC-Adresse oder E-Mail)
    \item Vertrauliche und Anmeldeinformationen (z.B. Login-Passwort-Kombinationen, API-Token oder digitale Zertifikate)
    \item Finanzinformationen (z.B. Bankkontodaten, Kreditkarteninformationen oder Verbrauchsdaten)
\end{itemize}
Die Kategorisierung und der Detailgrad können je nach Unternehmen variieren.

Neben den Kategorien muss auch der Kontext beachtet werden, in dem eine Information verwendet wird. Denn der Kontext hat direkten Einfluss auf die Sensibilität. Pogiatzis und Samakovitis \cite{Pogiatzis.2020} leiten vier verschiedene Kontextklassen ab, die auf der Bedeutung, der Interaktion, der Priorität und Präferenz basieren, die mit jeder Information verbunden sind.

\begin{itemize}
    \item Der semantische Kontext wird auf Grundlage der semantischen Bedeutung eines Begriffs gebildet. Die semantische Bedeutung einer Sequenz wirkt sich zum Beispiel auf ihre Sensibilität aus.
    \item Im Kontext der Akteure wird die Sensibilität von Daten abhängig von den Akteuren, die an der Informationsübermittlung beteiligt sind, betrachtet. Die Sensibilität wird durch die Beziehung zwischen den beteiligten Akteuren bestimmt. Zum Beispiel ist der Austausch von Gesundheitsinformationen zwischen Patient und Arzt nicht sensibel, außerhalb dieser Gruppe von Akteuren jedoch schon.
    \item Der zeitliche Kontext bezieht sich auf die Priorität der Informationen, die die Bedeutung des Begriffs beeinflussen. Eine Zeichenfolge, die als Passwort eingegeben wird, gilt bspw. als vertraulicher, als wenn sie als Benutzername eingegeben wird.
    \item Der Selbstkontext wird durch die persönlichen Präferenzen des Nutzers in Bezug auf seine Privatsphäre bestimmt. Zum Beispiel kann eine Person ihre ethnische Herkunft als vertrauliche Information betrachten, eine andere nicht.
\end{itemize}
Auch hier können verschiedene kontextuelle Kategorien unterschieden oder definieren werden. Sie sind zudem nicht immer klar trennbar und schließen sich nicht gegenseitig aus. Ein oder mehrere Kontexte können sich gleichzeitig unterschiedlich auf die Sensibilität auswirken. Manche Daten können jedoch auch unabhängig vom Kontext vertraulich sein, wie z.B. Passwörter oder Kreditkartennummern.

% Daten Klassen
Die Einteilung von Daten in verschiedene Geheimhaltungsklassen ist ein häufig verwendetes Verfahren in militärischen und behördlichen Anwendungen. Militärische Anwendungen verwenden dabei Begriffe wie \glqq eingeschränkt\grqq , \glqq vertraulich\grqq, \glqq geheim\grqq und \glqq streng geheim\grqq \cite{Landwehr.1984}. So ist es möglich, sensible Daten noch präziser in Vertraulichkeitsstufen zu unterteilen.

% Format -> structured, unstructured
Außerdem wird die Klassifizierung auch von der Struktur der Daten beeinflusst. Mehr als 80\% der Daten im Internet bestehen aus unstrukturierten Daten \cite{Allahyari.2017}. Unstrukturierte Daten beziehen sich in der Regel auf Informationen, die nicht in einer relationalen Datenbank gespeichert sind. Folglich gibt es kein vordefiniertes Datenmodell und die Struktur ist unregelmäßig oder unvollständig. Selbst Datenformate wie CSV, JSON oder XML, die einige organisatorische Eigenschaften haben, verfügen in der Regel nicht über ein klar definiertes Datenmodell. Im Vergleich zu strukturierten Daten ist es schwieriger, unstrukturierte Daten abzurufen, zu analysieren und zu speichern. Während Menschen unstrukturierte Daten leicht verarbeiten können, haben Maschinen oft Schwierigkeiten damit \cite{Guo.2021}.

Die Herausforderung bei der Datenklassifizierung besteht daher darin, die Kategorien, den Kontext, die Vertraulichkeitsstufen und die Struktur der Daten zu berücksichtigen.

\subsubsection{Daten Zustand}
% Stati von Daten (in Rest, in Motion etc.)

\subsubsection{Manuelle Datenklassifizierung}
% Warum manuell

% warum schwierig (aufwändig, fehlerbehaftet, Big Data)

\subsubsection{Automatische Datenklassifizierung}
% warum automatisiert

% wie automatisiert

% content vs. context statt rule/model/context

\paragraph{content-based}

\paragraph{context-based}



\subsection{Anwendung in der Cloud Security}
% wofür ist das dann gut

\subsubsection{Labeling für andere Maßnahmen}
% verschlüsselung
% Access Rights
% Versenden

\subsubsection{Anwendungsbeispiele}
% BrowserFlow
% DocGuard?

\section{Methoden der automatischen Datenklassifizierung}
% Warum manuell
% TODO: ggf. zusammenfassen
Viele Studien in der Literatur haben Klassifizierungsmethoden verwendet, um die Datensicherheit in der Cloud zu gewährleisten. Die vorgeschlagenen Lösungen lassen sich in zwei Klassen einteilen: die manuelle Klassifizierung, die vom Nutzer festgelegt wird, und die automatische Klassifizierung, bei der ein Algorithmus zum Einsatz kommt.

Die manuelle Datenklassifizierung ist trotz Fortschritte bei automatisierten Technologien eine gängige Methode. In einigen Fällen, wie bei der Klassifizierung von Informationen wie geistigem Eigentum oder Geschäftsgeheimnissen, bleibt die manuelle Klassifizierung erforderlich. Menschen sind in der Lage, leicht die Kategorien, Kontexte, Strukturen und Zustände der Daten ganzheitlich zu berücksichtigen. Außerdem kann die manuelle Klassifizierung für kleinere Unternehmen kostengünstig sein \cite{Divadari.2023}\cite{Alsuwaie.2021}.

% warum schwierig (aufwändig, fehlerbehaftet, Big Data)
Die manuelle Klassifizierung von Daten ist jedoch sehr anfällig für menschliche Fehler und Inkonsistenzen. Die Subjektivität der Menschen kann zu inkonsistenten Klassifizierungen führen, was die Genauigkeit und Zuverlässigkeit der Sicherheitsmaßnahmen beeinträchtigen kann. Zudem kann eine unzureichende Genauigkeit zu unvollständigen oder falschen Klassifizierungen führen. Im Normalfall werden Daten bei ihrer Erstellung klassifiziert, doch sie können sich im Laufe der Zeit ändern, wodurch die ursprüngliche Klassifizierung veraltet und nicht mehr richtig sein kann. Die manuelle Einordnung von großen Datenmengen kann sehr zeitaufwändig und arbeitsintensiv sein und ab einer bestimmten Größe nicht mehr manuell verarbeitet werden. Manuelle Prozesse können die Anpassungsfähigkeit und schnelle Reaktion auf sich ändernde Geschäftsanforderungen reduzieren \cite{Venhorst.2019}.

Aufgrund der wachsenden Datenmengen und der zunehmenden Komplexität der Informationssicherheitsanforderungen erscheint die automatisierte Datenklassifizierung häufig als effizientere Lösung, die eine genauere und konsistentere Identifizierung sensibler Daten ermöglicht.

% ###########################################
% warum automatisiert
Der Prozess, bei dem maschinelle Algorithmen und Technologien verwendet werden, um Daten automatisch zu identifizieren, zu kategorisieren und entsprechend ihres Sensitivität zu klassifizieren, wird als automatische Datenklassifizierung bezeichnet. Diese Methode verwendet maschinelles Lernen, Mustererkennung oder künstliche Intelligenz, um Daten zu analysieren und automatisch geeignete Klassifizierungen zuzuweisen.
Die aktuellen Techniken in der Data Leakage Prevention können allgemein in zwei Kategorien eingeteilt werden: inhaltsbasierte Analyse und kontextbasierte Analyse. Methoden, die auf dem Inhalt basieren, untersuchen den Inhalt von Daten anhand von Merkmalen sensibler Informationen wie regulären Ausdrücken und Datenfingerabdrücken. Inhaltsbasierte Methoden verwenden vorhersehbare Muster wie z.B. IP-Adressen oder E-Mail-Adressen, um sensible Daten zu erkennen. Kontextbasierte Techniken identifizieren vertrauliche Daten anhand von Merkmalen im Zusammenhanf mit den überwachten Daten. Der kontextbasierte Ansatz ist damit effektiver für vertrauliche Daten ohne vorhersehbare Muster \cite{Guo.2021}\cite{Gugelmann.2015}\cite{Kuzina.2023}. Um sensible Informationen umfassender und genauer zu extrahieren, sollten daher verschiedene Methoden angewendet werden. Die Tabelle \ref{t:methoden} zeigt die am häufigsten verwendeten Methoden in der Literatur. Die meisten kontexbasierten Ansätze kombinieren inhaltsbasierte und kontextbasierte Methoden, um die Vorteile beider Kategorien zu nutzen und die Genauigkeit der Klassifizierung zu verbessern.

%Tabelle mit content-based vs. context-based
\begin{table}[htbp]
  \normalsize
  \caption{Methoden der automatischen Datenklassifizierung. Quelle: eigene Darstellung.}
  \label{t:methoden}
  \begin{center}
    \begin{tabulary}{15cm}{|L|L|}
      \hline
      \textbf{Kategorie} & \textbf{Methode} \bigstrut  \\
      \hline
      \hline
      \multirow{5}{*}{inhaltsbasiert} & regelbasierte Methoden \bigstrut[t]     \\
      & Data Fingerprinting               \\
      & kNN              \\
      & Boosting \\
      & Clusteranalyse \\
      \hline
      \multirow{2}{*}{inhalts- und kontextbasiert} & CASSED  \bigstrut[t] \\
      & BERT-BiLSTM     \\
      \hline
    \end{tabulary}
  \end{center}
\end{table}

% ###########################################
\subsection{Klassifizierung mit manueller Definition}

% ###########################################
\paragraph{regelbasierte Methoden}
% rule-based/regular-matching/Dictionary
Eine einfache und häufig angewendete Technik zur automatischen Datenklassifizierung basiert auf einem Wörterbuch oder einer Regel, die im Wesentlichen den gegebenen Text mit einer Liste vordefinierter regulärer Ausdrücke und Schlüsselwörter abgleicht. Diese Methode verwendet vordefinierte Regeln und Bedingungen, um bestimmte Datentypen oder Muster automatisch zu identifizieren und zu klassifizieren. Diese Regeln können auf verschiedenen Merkmalen wie Schlüsselwörtern, Mustern, Dateiformaten oder spezifischen Attributen wie Datumsangaben basieren \cite{Ong.2017}.
Die Identifikation von Kreditkartennummern in Textdokumenten ist ein Beispiel für die Verwendung regelbasierter Methoden. Zur Erkennung kann ein regulärer Ausdruck verwendet werden, der die Zeichenfolge nach dem Muster einer Kreditkartennummer definiert. Muster in regulären Ausdrücken umfassen meistens normale Zeichen mit wörtlicher Bedeutung und Metazeichen, um ein Erkennungsmuster zu bilden \cite{Alneyadi.2016}.
Regeln können sich auch auf bestimmte Schlüsselwörter oder Phrasen beziehen, die auf personenbezogene Informationen wie \glqq Sozialversicherungsnummer\grqq \:oder \glqq vertraulich\grqq \:hinweisen können.
Der Vorteil regelbasierter Methoden liegt in ihrer klaren Struktur und der Möglichkeit, spezifische Anforderungen und Richtlinien der Organisation abzubilden. Sie ermöglichen eine präzise und konsistente Klassifizierung von Daten gemäß vordefinierten Sicherheitsstandards. Da die Klassifizierung auf klaren, vorher festgelegten Regeln basiert, ermöglichen regelbasierte Ansätze auch eine gewisse Transparenz und Nachvollziehbarkeit.
Jedoch können regelbasierte Methoden bei der Verarbeitung komplexer und sich verändernder Datenmuster weniger nützlich sein, da die Regeln schnell unpraktisch werden, wenn Datenformate, Kontext, Wortvariationen und Abkürzungen kombiniert werden müssen. Der Einsatz dieser Methode erfordert auch ein gut definiertes und gepflegtes Wörterbuch und Regelsatz. AAußerdem wird der semantische Kontext der Wörter bei einem reinen Textabgleich nicht berücksichtigt, was zu einer geringen Genauigkeit der Klassifizierung führen kann \cite{Ong.2017}.
Trotzdem bieten regelbasierte Ansätze eine grundlegende und robuste Methode zur Sicherstellung einer konsistenten Datenklassifizierung.

% ###########################################
\paragraph{Data Fingerprinting}
Data Fingerprinting oder auch Document Fingerprinting erstellt eindeutige Fingerabdrücke für bestimmte Datenfragmente oder ganze Dateien. Diese Fingerabdrücke sind eindeutige Identifikatoren für die entsprechenden Daten und werden genutzt, um sensible Daten zu identifizieren und automatisch zu klassifizieren. Eindeutige Fingerabdrücke für Wörter, Sätze oder ganze Dateien werden mithilfe von Wortmustern aus regulären Ausdrücken oder vordefinierten Wörterbüchern erstellt und als Vorlage für sensible Daten verwendet. Diese Fingerabdrücke können dann verwendet werden, um Fingerabdrücke von nicht-klassifizierten Daten zu vergleichen und zu klassifizieren.
Häufig werden Hash-Funktionen wie MD5 oder SHA1 verwendet, um Datenfingerprints zu erstellen, die eine algorithmisch generierte Zeichenfolge fester Größe für die Daten darstellen. Die Hashes von zwei Dateien unterscheiden sich jedoch, sobald nur ein Zeichen verändert wurde \cite{Alneyadi.2016}. Ein weiterer Ansatz ist deshalb das \glqq Fuzzy-Hashing\grqq. Hierbei werden die Daten in Blöcken verarbeitet, wodurch die Hash-Ausgabe bei ähnlichen Daten größtenteils übereinstimmende Blöcke enthält. So kann die prozentuale Ähnlichkeit mithilfe einer mathematischen Vergleichsfunktion bestimmt werden \cite{Shu.2015}.

Sowohl im Speicher als auch bei Netzwerkübertragungen oder bei Verwendung kann das Data Fingerprinting sensible Daten erkennen. Der Fingerabdruck ist in manchen Fällen jedoch keine zuverlässige Methode. Die Klassifizierung funktioniert nicht, wenn die Daten verschlüsselt oder passwortgeschützt sind oder der Inhalt nicht eindeutig mit dem Fingerabdruck übereinstimmt. Außerdem ist es notwendig, dass die Vorlagen kontinuierlich aktualisiert werden und der Ansatz kann bei großen Datenmengen ressourcenintensiv sein.


% ###########################################
\subsection{Klassifizierung mit maschinellem Lernen}
% Ali Abdulsattar Jabbar, Wesam Sameer Bhaya 2023 – Security of private cloud using
% nur Text?
% Definition
% Datenaufbereitung
% wie wird Text in Algorithmen überhaupt dargestellt -> Vektoren
Die Datenklassifizierung ist eine Technik des maschinellen Lernens, mit der die Klasse der nicht klassifizierten Daten vorhergesagt wird. Dabei werden die Methoden in zwei Kategorien eingeteilt: überwachtes Lernen und unüberwachtes Lernen. Für überwachtes Lernen sind Testdaten mit bereits zugeteilten Klassen vordefiniert. So kann das Modell sein Ergebnis mit der Ziel-Klasse vergleichen und entsprechend von den Fehlern lernen. Beim unüberwachten Lernen sind keine Klassen definiert, sondern die Klassifizierung der Daten erfolgt automatisch. Ein unüberwachter Algorithmus sucht nach Mustern und Ähnlichkeiten zwischen den Elementen cite{Frochte.2018b}.

Für die meisten Methoden des maschinellen Lernens ist es erforderlich, dass die Daten vorab bereinigt oder vorbereitet werden. Die einzelnen Schritte können dabei je nach Methode variieren. Der Datensatz wird meistens von Stop-Wörtern bereinigt, die keinen oder nur wenig Kontext liefern wie bspw. 'und' oder 'der'. Klassische Methoden sind in der Textverarbeitung Stemming und Lemmatization. Beim Stemming werden die Worte der Eingabe auf ihren Wortstamm zurückgeführt, bei der Lemmatization werden ähnliche oder inhaltlich gleiche Begriffe vereinheitlicht.
Dadurch wird die grammatikalische Komplexität des Inputs reduziert und die Datenqualität optimiert. Für Methoden wie neuronale Netze müssen Eingabedaten zusätzlich noch in Token und Vektoren umgewandelt werden. Bei der Tokenization wird der Text ist einzelne Bestandteile, sogenannte Tokens aufgeteilt werden. Meistens sind das einzelne Worte. Die Tokens werden anschließend in Vektoren umgewandelt, die den jeweiligen Token im Modell repräsentiert. Für die Vektorisierung gibt es verschiedene Algorithmen, um Informationen wie inhaltlich ähnliche oder zusammenhängende Tokens zu behalten \cite{Kamath.2019}.


% ###########################################
\paragraph{k-NN}
% K-NN (k Nearest Neighbors)
Zardari, Jung et al. \cite{Zardari.2014} waren die ersten, die eine Methode aus dem maschinellen Lernen zur Datenklassifizierung verwendet haben, um im Cloud Computing die Datensicherheit zu verbessern. Sie verwendeten die k-Nearest Neighbor-Methode (k-NN), um Daten in der Cloud als sensibel und nicht-sensibel zu klassifizieren. k-NN ist eine Methode aus dem überwachten maschinellen Lernen und wird häufig zur Klassifizierung, Mustererkennung und Schätzung verwendet. Es handelt sich um einen sogenannten Instanz-basierten Algorithmus, der darauf basiert, dass ähnliche Dinge in der Regel nah beieinander liegen. Die k-NN-Methode verwendet die Nachbarn eines Datenpunkts, um Vorhersagen zu treffen oder Klassenzuweisungen zu machen. Der Algorithmus beginnt mit einem Datensatz, der aus Beispielen mit bekannten Klassen oder Werten besteht. Die Wahl des 'k' bestimmt die Anzahl der nächsten Nachbarn. Um die Ähnlichkeit zwischen den Datenpunkten zu bestimmen, wird ein Distanzmaß wie der euklidische Abstand oder die Manhattandistanz verwendet. Sie messen den Abstand zwischen den Merkmalsvektoren der Datenpunkte. Für einen gegebenen Datenpunkt werden die k nächsten Nachbarn basierend auf dem berechneten Distanzmaß aus dem Datensatz ausgewählt. Bei einer Klassifikation werden die Klassen der ausgewählten k Nachbarn betrachtet und die Mehrheitsklasse für den gegebenen Datenpunkt verwendet \cite{Frochte.2018b}.

In \cite{Zardari.2014} hat die Einteilung der Daten in zwei Klassen gut funktioniert. Doch die Wahl des passenden 'k' ist bei diesem Algorithmus entscheidend. Ein zu kleines 'k' kann dazu führen, dass potenziell passende Datenpunkte ausgeschlossen werden und ein zu großes 'k' führt zu einer zu groben Klassifizierung. Aufgrund der Verwendung aller Trainingsdaten ist zudem die Rechenkomplexität bei diesem Algorithmus hoch, da bei jeder Vorhersage der Abstand berechnet und die gesamten Trainingsdatendistanzen sortiert werden müssen. Außerdem ist die Mehrheitsentscheidung bei der Klassifizierung nicht immer die optimale Methode, da je nach Anzahl der Nachbarn die Abstände stark variieren können und trotzdem immer alle gewählten Nachbarn berücksichtigt werden.

% ###########################################
\paragraph{Boosting}
Ensemble Learning ist ein Begriff aus dem maschinellen Lernen und beschreibt das Zusammenschalten von mehreren Methoden, um das Modellergebnis zu verbessern. Der Ansatz beim Boosting ist, mehrere schwache Modelle zu einem starken Modell zusammenzusetzen \cite{Frochte.2018c}.
Kaur, Zandu \cite{Kaur.2016} schlagen für die Klassifizierung von sensiblen Daten eine neue Boosting-Architektur vor. Als Klassifikator wird eine Kombination aus dem Naive Bayes Klassifikator und AdaBoost verwendet. Der Naive Bayes Algorithmus ist eine Klassifizierungsmethode, die auf dem Bayes Theorem beruht. Die Grundlage ist die Annahme, dass das Auftreten eines Merkmals unabhängig mit dem Auftreten eines anderen Merkmals innerhalb der Klasse ist. Das Bayes Theorem ist eine Formel zur Berechnung der bedingten Wahrscheinlichkeit ${P(A|B)}$. Diese bedingte Wahrscheinlichkeit lässt sich mit der Formel \ref{e:naive-bayes} berechnen. Zur Klassifizierung eines Merkmals wird mit der Formel für jede Klasse berechnet, mit welcher Wahrscheinlichkeit das Merkmal zu dieser Klasse gehört. Gewählt wird schließlich die Zuordnung mit der höchsten Wahrscheinlichkeit \cite{Frochte.2018d}.

\begin{equation}
  \label{e:naive-bayes}
  P(A|B) = \frac{P(B|A) * P(A)}{P(B)}
\end{equation}

Im Ansatz von \cite{Kaur.2016} wird dieser Klassifikator mit AdaBoost, Abkürzung für adaptive Boosting, optimiert. AdaBoost kombiniert mehrere schwache Klassifikatoren zu einem starken Klassifikator. Dabei werden iterativ mehrere Klassifikatoren hinzugefügt und der Datensatz stetig neu gewichtet, damit sich der nächste Klassifikator auf die Fehler des vorherigen Klassifikators konzentriert \cite{Frochte.2018c}.

In der neuen Klassifikationsmethode von \cite{Kaur.2016} wird zuerst der Trainingsdatensatz in eine bestimmte Anzahl an Datensätzen aufgeteilt. Anschließend werden die einzelnen Teildatensätze schrittweise verarbeitet. Jeder Teildatensatz enthält eine Menge an Daten-Tupeln. Zu Beginn erhält jedes Tupel die gleiche Gewichtung. Dann wird das hybride Klassifizierungsmodell aus Naive Bayes und AdaBoost mit dem Teildatensatz trainiert. Anschließend werden bei den Tupeln die Gewichte aktualisiert, je nachdem ob sie richtig oder falsch klassifiziert wurden. Nach den Durchgängen aller Teildatensätze ergibt sich dadurch ein Set an Klassifikator-Modellen mit jeweiligen Gewichtungen, die in Kombination eine genaue Vorhersage für eine Klasse treffen.

Kaur, Zandu \cite{Kaur.2016} zeigten, dass ihre vorgeschlagene Methode mit 94,2529\% Genauigkeit deutlich besser klassifiziert als der k-NN-Algorithmus, der nur eine Genauigkeit von 51,7241\% hatte.

% ###########################################
\paragraph{Clusteranalyse}
% auf Basis von wenigen manuell klassifizierten Daten können Cluster gebildet werden
Die Clusteranalyse ist eine Datenanalysetechnik aus dem Bereich maschinellen Lernens. Clustering ist eine unüberwachte Lernmethode, die Muster in Eingabedaten ohne vordefinierte Zielwerte erkennt. Beim Clustering werden unsortierte Informationen auf der Grundlage von Ähnlichkeiten, Mustern und Unterschieden gruppiert, ohne dass die Daten zuvor trainiert wurden. Im Zusammenhang mit der Klassifizierung sensibler Daten wird die Clusteranalyse auf die Daten in einem Unternehmen angewendet. Die daraus resultierenden Cluster enthalten dann ähnliche Dokumente nach einer Ähnlichkeitsmetrik wie dem euklidischen Abstand oder der Kosinusähnlichkeit \cite{Suyal.2014}.
Um die Clusteranalyse als Klassifikator zu nutzen, müssen vor der Analyse relevante Merkmale oder Attribute definiert werden, um sensible Daten zu identifizieren. Anschließend können die Cluster anhand der Merkmale klassifiziert werden.

Zwei gängige Clustering-Methoden sind das hierarchische Clustering und das partitionierende Clustering. Hierarchisches Clustering wird in der Regel für dsa Clustering von Texten verwendet, wobei jedes Dokument auf der Grundlage seiner Ähnlichkeit schrittweise in einen vordefinierten Cluster zusammengeführt wird. Durch diesen Prozess entsteht eine Clusterhierarchie, die als Baumstruktur, das sogenannte Dendrogramm, dargestellt werden kann. Abbildung \ref{f:clustering_daten} zeigt einen Beispieldatensatz mit den Objekten a bis f. In diesem Datensatz liegen die Objekte b und c sowie d und e sehr nahe beieinander. Der Clustering-Algorithmus fasst dann nach und nach die Objekte mit dem geringsten Abstand zusammen,  gefolgt von den nächstgelegenen Objekten oder Clustern, bis der gesamte Datensatz gruppiert ist. So entsteht ein Baum wie in Abbildung \ref{f:clustering_baum}, bei dem die Blätter Cluster darstellen, die nur ein einzelnes Objekt aus dem Datensatz enthalten, und die Wurzel einen einzelnen Cluster, der alle Objekte enthält. Die Kanten zwischen den Knoten enthalten außerdem ein Attribut, das den Abstand zwischen den beiden Clustern angibt. Je nach gewünschter Anzahl von Clustern können die Cluster auf einer bestimmten Ebene des Baumes verwendet werden \cite{Suyal.2014}.
% TODO: Distanz Algorithmus im Detail darstellen?

\begin{figure}[htbp]
  \centering
  \begin{subfigure}[b]{0.47\linewidth}
    \includegraphics[width=\linewidth]{\figdir/clustering_daten}
    \caption{Datensatz. Quelle: In Anlehnung an \cite{Bonthu.2023}.}
    \label{f:clustering_daten}
  \end{subfigure}
  \hfill
  \begin{subfigure}[b]{0.47\linewidth}
    \includegraphics[width=\linewidth]{\figdir/clustering_dendrogramm}
    \caption{Dendrogramm. Quelle: In Anlehnung an \cite{Bonthu.2023}.}
    \label{f:clustering_baum}
  \end{subfigure}
\end{figure}

Aufgrund seiner Einfachheit und Flexibilität wird hierarchisches Clustering häufig verwendet und bietet den Vorteil, dass jede Art von Ähnlichkeitsmessung durchgeführt werden kann. Außerdem bietet dieses Verfahren eine detaillierte Darstellung der Clusterstruktur, wodurch unterschiedliche Granularitätsstufen von Clustern untersucht werden können.

% TODO: Partitionierende Clustering k-mean
Beim partitionierenden Clustering wird ein Datensatz in eine vorab definierte Anzahl von Clustern eingeteilt. Jeder Datenpunkt gehört zu einem bestimmten Cluster und das Ziel besteht darin, möglichst viele Datenpunkte in die Cluster zu verteilen und dabei die Ähnlichkeit zwischen den Clustern zu minimieren. DAs am weitesten verbreitete Verfahren ist der k-means-Algorithmus. Das 'k' steht für die Anzahl an zu definierenden Clustern und 'means' für den Mittelwert, also das Zentrum des Clusters. Zu Beginn muss die Anzahl der Cluster bestimmt werden. Dies kann sich z.B. daran orientieren, in welche Vertraulichkeitsstufen die Daten eingeteilt werden sollen. Anschließend werden für die Cluster initial jeweils zufällig ein Cluster-Mittelpunkt, auch Centroid genannt, gewählt. Dann wird für jeden Datenpunkt der Abstand zwischen dem Punkt und den Cluster Centroids berechnet. Der Punkt wird dem jeweiligen Cluster zugeordnet, welcher am nächsten ist und die Cluster sind initial befüllt. Nun folgen Schritte, die sich solange wiederholen, bis sich die Cluster nicht mehr ändern. Zuerst wird für jedes Cluster aus den Datenpunkten ein neuer Mittelwert bestimmt, der den neuen Centroid darstellt. Dann werden alle Datenpunkte anhand ihrer Distanzen zu den neuen Zentren neu zugeordnet \cite{Suyal.2014}.

Der k-mean Algorithmus ist beliebt, da er einfach ist, nur eine kleine Anzahl an Iterationen benötigt und parallel berechnet werden kann. Allerdings ist das Ergebnis des Algorithmus stark abhängig von der Wahl des 'k' und der initialen Cluster \cite{Suyal.2014}.


% ###########################################
\paragraph{Context-based Approach for Structured Sensitive Data Detection}
Kužina, Petric et al. \cite{Kuzina.2023} sahen eine große Herausforderung darin, sensible Daten in strukturierten Datenbanken zu klassifizieren. Das Problem besteht darin, die einzelnen Spalten einer Datenbanktabelle zu durchsuchen und zu bestimmten, ob sie sensible Daten enthalten und welche Arten sensibler Daten vorhanden sind. Dafür muss der Inhalt der Tabellen-Zelle interpretiert werden und der Kontext der umgebenden Zellen berücksichtigt werden. Bisherige Ansätze verwendeten dafür stark regelbasierte Methoden, deren Grenzen bei einer großen Menge an verschiedenen Datentypen schnell erreicht wurden. Außerdem können sie nur begrenzt Kontext und Semantik miteinbeziehen. Um dieses Problem zu lösen, entwickelten
Kužina, Petric et al. \cite{Kuzina.2023} eine neue Methode namens 'Context-based Approach for Structured Sensitive Data Detection' (CASSED). Dabei stellen sie einen Spaltenkontext durch die Kombination von Spaltenmetadaten und Zellwerten her, der in einen einzelnen Input-Vektor umgewandelt wird. Mit diesem Input-Vektor wird anschließend das BERT-Modell für die Klassifizierung verwendet.

BERT steht für 'Bidirectional Encoder Representations from Transformers' und ist ein von Google entwickeltes Open-Source Framework zur Erstellung von Transformer-basierten Natural-Language-Processing-Modellen. BERT ist darauf spezialisiert, kontextuelle Zusammenhänge und Beziehungen zwischen Wörtern zu erfassen. Transformer-basierte Modelle nutzen einen sogenannten Selbstaufmerksamkeitsmechanismus, indem die Beziehung eines Wortes mit jedem anderen Wort in einem Satz bestimmt wird. Außerdem enthalten sie mehrere Encoder- und Decoder Schichten, die einen Text lesen und versuchen, das nächste Wort vorherzusagen, sowie voll-vernetzte neuronale Netze. BERT nutzt Transformer, einen Aufmerksamkeitsmechanismus und nur Encoder-/Decoder-Schichten. BERT verarbeitet Texteingaben bidirektional, indem die Sequenzen sowohl von Anfang als auch vom Ende her analysiert werden, um ein besseres Verständnis für die kontextuellen Beziehungen der Wörter zu bekommen.

Im ersten Schritt erfolgt die Umwandlung der Spalten in Input-Vektoren. Dabei werden die Spaltenüberschrift zusammen mit mehreren Zellwerten derselben Spalte als Tokens dargestellt und durch Trennzeichen getrennt. In Abbildung \ref{f:input} ist eine Umwandlung einer Spalte in einen Input-Vektor dargestellt. Dabei wird die Spaltenüberschrift mit einem Punkt zur ersten Zelle getrennt, während die Zellen zueinander mit einem Komma getrennt werden. So wird dem Modell zusätzliche Informationen geliefert, dass diese Werte unterschiedlich behandelt werden sollten. BERT kann nur eine Anzahl von maximal 512 Tokens als Input-Vektor verarbeiten. Deshalb müssen größere Input-Vektoren noch aufgeteilt werden.

\begin{figure}[htbp]
  \centering
  \includegraphics[width=0.8\linewidth]{\figdir/cassed_input.png}
  \caption{Beispiel einer Umwandlung einer Spalte in einen Input-Vektor. Quelle: \cite{Kuzina.2023}.}
  \label{f:input}
\end{figure}

Zur Klassifizierung erzeugt der Decoder von BERT für jede mögliche Klasse eine nicht normalisierte Vorhersage, die anschließend über alle Spaltenteile gemittelt wird, in die die Spalte aufgeteilt wurde. Auf diese Vorhersage-Werte wird eine Sigmoidfunktion angewendet, um normalisierte Wahrscheinlichkeiten für jede Klasse zu erzeugen.

Zusätzlich zum BERT-Modell enthält der CASSED-Ansatz eine regelbasierte Schicht, die strukturierte Formate wie E-Mails oder Sozialversicherungsnummern mittels regulären Audrücken klassifiziert. Außerdem wird ein Wörterbuch für bekannte sensible Daten oder Merkmale von Geschäftsgeheimnissen eingesetzt. Auch diese Schicht ermittelt für jede Klasse eine mögliche Wahrscheinlichkeit. Diese und die Wahrscheinlichkeiten der BERT-Schicht werden kombiniert zu einer Gesamt-Wahrscheinlichkeit pro Klasse. Anschließend werden die Klassen zugeteilt, deren Wahrscheinlichkeit einen bestimmten Schwellwert überschreiten. So ist es auch möglich, dass eine Spalte mehrere Klassen erhält. Abbildung \ref{f:cassed} stellt die Architektur des CASSED-Ansatzes dar.

\begin{figure}[htbp]
  \centering
  \includegraphics[width=0.8\linewidth]{\figdir/cassed_model.png}
  \caption{Überblick über die CASSED Methode. Quelle: \cite{Kuzina.2023}.}
  \label{f:cassed}
\end{figure}

Im Vergleich zu anderen kontextbasierten Klassifizierungsmethoden erzielt die CASSED Methode deutlich bessere Ergebnisse. Allein durch den reinen Einsatz von BERT schneidet das Modell besser ab und ist durch die regelbasierte Schicht sogar noch präziser.

Einen ähnlichen Ansatz verfolgen auch Guo, Liu et al. \cite{Guo.2021} mit ihrem Ansatz 'Exsense'. Auch sie verwenden zwei Schichten: eine inhaltsbasierte Analyse mit regulären Ausdrücken und eine kontextbasierte Analyse mit einem BERT-BiLSTM-Attention-Modell. Dabei wird das BERT-Modell kombiniert mit einem Bidirectional Long Short-Term Memory Modell (BiLSTM). BiLSTM ist ein rekurrentes neuronales Netzwerk, das für die Verarbeitung von sequenziellen Daten entwickelt wurde und gut darin, Abhängigkeiten in langen Sequenzen zu erfassen. Im BERT-BiLSTM-Modell wird die bidirektionale Kontextrepräsentation von BERT als Eingabe für das BiLSTM verwendet. Auch dieser Ansatz erzielte sehr gute Ergebnisse zur Klassifizierung von sensiblen Daten unter Berücksichtigung des Kontextes.


% ###########################################
% \paragraph{TF-IDF}
%'Term Frequency' und 'Inverse Document Frequency' (TF-IDF) ist eine häufiger verwendete Technik, bei der die Bedeutung eines Begriffs direkt proportional zur Häufigkeit des Auftretens des Begriffs in einem Dokument ist und umgekehrt proportional zur Häufigkeit des Auftretens des Begriffs im gesamten Dokument ist. Diese Technik wird häufig in verschiedenen Bereichen wie Data Mining oder Textklassifizierung eingesetzt.

% ###########################################
\subsection{Anwendung in der Cloud Security}
% wofür ist das dann gut

\subsubsection{Labeling für andere Maßnahmen}
% verschlüsselung
% Access Rights
% Versenden
% fingerprint

% BrowserFlow
% DocGuard?


\section{Ausblick}
Diese Arbeit gibt einen Überblick über verschiedene Ansätze im Bereich der Cloud Data Leakage Prevention mit Methoden der automatischen Datenklassifizierung. Zunächst wird die Problematik der versehentlichen Offenlegung von Cloud-Daten analysiert. Anschließend werden verschiedene Ansätze zur Prävention von Datenlecks in der Cloud vorgestellt mit Schwerpunkt auf der Nutzung von Cloud-DLP-Systemen. Dabei hat sich herausgestellt, dass die Erkennung sensibler Daten beeinflusst wird von den verschiedenen möglichen Eigenschaften von Daten, wie Kontext, Kategorien, Vertraulichkeitsstufen, Struktur und Zuständen. Im Fokus steht die automatische Datenklassifizierung, die als entscheidendes Element für den Schutz sensibler Informationen betrachtet wird. Hierbei werden verschiedene Methoden betrachtet, darunter die Klassifizierung mit manueller Definition, die regelbasierte Methoden und Data Fingerprinting umfasst. Zudem wird die Anwendung von maschinellem Lernen in der automatischen Datenklassifizierung untersucht, wobei Algorithmen wie k-NN, Boosting, Clusteranalyse und ein kontextbasierter Ansatz für die Erkennung strukturierter sensibler Daten betrachtet wurden. Mit der reinen Klassifizierung von sensiblen Daten sind diese jedoch noch nicht geschützt. Deshalb wurden im letzten Kapitel noch einige Schutzmaßnahmen betrachtet, die auf klassifizierten Daten basieren.

Besonders die manuelle Datenklassifizierung und auch die Klassifizierung von manuellem Zutun werden immer risikoreicher. Bei Verwendung von manuellen Sicherheitsrichtlinien, besteht die Gefahr, dass nicht alle wichtigen und sensiblen Daten einbezogen werden und die Richtlinien nicht stetig aktualisiert werden. Außerdem bedeutet manueller Aufwand, dass dies durch geschultes Sicherheitspersonal durchgeführt werden muss. Doch diese Mitarbeiter sind auch nicht allwissend über die gesamten Daten innerhalb des Unternehmens.
Anhand der Literatur-Recherche hat sich gezeigt, dass sich die automatische Datenklassifizierung bewährt hat und vor allem daran geforscht wird, inwiefern die Klassifizierung verbessert und effizienter gemacht werden kann. Insbesondere der verstärkte Einsatz von Cloud-Technologien stellt die Data Leakage Prevention vor neue Herausforderungen. Die Verteilung von Daten auf verschiedene Medien, der schnelle Zugriff und die großen Datenmengen sind nur einige davon. Trotz der herausragenden Ergebnisse des Einsatzes von Methoden des maschinellen Lernens, stellt es Unternehmen auch vor neue Herausforderungen. Die Entwicklung eines maschinellen Lernmodells zur Erkennung sensibler Daten hängt stark von der Verfügbarkeit realer Datensätze mit vertraulichen Daten ab. Allerdings sind diese Datensätze nicht immer für die Öffentlichkeit zugänglich und enthalten möglicherweise auch nicht alle für ein Unternehmen nötigen Eigenschaften. In Datenschutz-relevanten Bereichen werden deshalb oft synthetische Datensätze verwendet. Diese Datensätze können sowohl zum Trainieren des Modells, als auch zur Erweiterung eines realen Datensatzes verwendet werden \cite{Kuzina.2023}.

Die vorgestellten Ansätze bieten vielversprechende Möglichkeiten für die Verbesserung der Cloud-Sicherheit. Technologische Fortschritte bieten der Cloud-Sicherheit neue Möglichkeiten, jedoch auch neue Herausforderungen.

% references ieeetr
\bibliographystyle{ieeetr}
\bibliography{ref}

\end{document}
