\documentclass[conference]{IEEEtran}
\IEEEoverridecommandlockouts
% The preceding line is only needed to identify funding in the first footnote. If that is unneeded, please comment it out.
\usepackage{cite}
\usepackage{amsmath,amssymb,amsfonts}
\usepackage{algorithmic}
\usepackage{graphicx}
\usepackage{textcomp}
\usepackage{xcolor}
\usepackage{balance}

\usepackage[utf8]{inputenc}
\usepackage[T1]{fontenc} % Trennen von Wörtern mit Umlauten
\usepackage{german} % Damit z.B. "Literatur" statt "References" da steht


\def\BibTeX{{\rm B\kern-.05em{\sc i\kern-.025em b}\kern-.08em
    T\kern-.1667em\lower.7ex\hbox{E}\kern-.125emX}}

\begin{document}

\title{Cloud Data Leakage Prevention mit Methoden der automatischen Datenklassifizierung}

\author{
    \IEEEauthorblockN{Anna Hamberger}
    \IEEEauthorblockA{\textit{Fakultät für Informatik} \\
        \textit{Technische Hochschule Rosenheim}\\
        Rosenheim, Germany \\
        anna.hamberger@stud.th-rosenheim.de}
}

\maketitle

% abstract & keywords
\begin{abstract}
    Die Nutzung von Cloud-Diensten wird immer beliebter, da sie große Datenmengen effizient speichern, schnellen Zugriff auf Ressourcen bieten und einen nahtlosen Datenaustausch ermöglichen. Allerdings steigt damit auch das Risiko, dass Daten in der Cloud ungewollt offengelegt werden. Das Problem der Datenverluste ist zu einer kritischen Herausforderung geworden, die die Entwicklung umfassender Systeme in diesem Bereich erforderlich macht. Ein System zur Verhinderung von Datenlecks (Data Leakage Prevention, DLP) konzentriert sich darauf, sensible Daten zu identifizieren und klassifizieren und Verstöße gegen Sicherheitsrichtlinien zu verhindern und zu melden. Ein wesentlicher Aspekt dieses Systems ist die automatische Klassifizierung von sensiblen Daten. Es gibt zwar manuelle Methoden, aber maschinelle Lernverfahren haben sich als äußerst effektiv erwiesen. Cloud-DLP-Systeme können auf der Grundlage der klassifizierten Daten verschiedene Sicherheitsmaßnahmen ergreifen und so den Schutz vor Datenlecks verbessern.


\end{abstract}

\begin{IEEEkeywords}
    component, formatting, style, styling, insert
\end{IEEEkeywords}

% chapters
\section{Einführung}
% Einleitung warum Cloud immer beliebter wird
Im Jahr 2022 gaben bereits 84\% der befragten 552 Unternehmen in Deutschland an, dass sie Cloud-Dienste in ihrem Unternehmen einsetzen \cite{KPMG.2022}. Cloud Computing hat sich in der Zeit der digitalen Transformation zu einem wichtigen Bestandteil der Informationsverarbeitung entwickelt. Die Nutzung von Cloud-Diensten wird immer beliebter, da sie die Möglichkeit zur effizienten Speicherung großer Datenmengen, schnellen Zugang zu Ressourcen und nahtlosen Datenaustausch bietet. Durch den Verbreitung von digitaler Technologie in der Gesellschaft und in Unternehmen werden immer mehr Daten geteilt und gesammelt. Um diese großen Datenmengen sammeln und verarbeiten zu können, nutzen Unternehmen die Vorteile von Cloud-Diensten. Die Möglichkeit, Daten in Echtzeit zu teilen, verbessert Geschäftsprozesse und erleichtert die Zusammenarbeit im Unternehmen \cite{Surianarayanan.2023b}. %18-27

% warum Datensicherheit
Da Informationen das wertvollste Gut eines Unternehmens sind, ist ihr Schutz von größter Bedeutung. Beim Sammeln von Daten ist ein Unternehmen zudem verpflichtet, sie vor Diebstahl, Verlust und Missbrauch zu schützen. Es gibt zahlreiche Datenschutzgesetze und -vorschriften, wie die EU-Datenschutz-Grundverordnung (DSGVO), um sensible Daten wie personenbezogene Daten zu schützen. Ziel dieser Vorschriften ist es, strengen Regeln für das Sammeln von Daten vorzugeben und der Einzelperson eine vergleichsweise hohe Kontrolle über ihre personenbezogenen Daten zu geben \cite{Kuzina.2023}. % 1
Unabhängig des Speicherorts besteht also das Risiko, dass die Datensicherheit verletzt wird.

% Relevanz Data Leakage Prevention
Eines der Hauptziele der Informationssicherheit ist die Verhinderung der Offenlegung von Daten gegenüber Unbefugten. Datenlecks können jedoch aufgrund der Notwendigkeit, auf Informationen zuzugreifen, diese zu teilen und zu nutzen, nicht immer verhindert werden. Diese Bedrohung kann von böswilligen Außenstehenden ausgehen, die versuchen, sensible Daten zu erhalten. Umgekehrt können auch interne Mitarbeiter eine Gefahr darstellen, wenn sie beabsichtigt oder unbeabsichtigt Informationen preisgeben \cite{Alneyadi.2016}. Bereits im Jahr 2018 haben Studien gezeigt, dass 53\% der befragten Unternehmen Insider-Angriffe in den letzten 12 Monaten bestätigten. Dabei sind Bedrohungen von innen häufig schwerwiegender als von außen, da sie meist schwieriger zu erkennen sind \cite{CATechnologies.2018}.
Die Offenlegung von sensiblen Daten kann erheblichen Schaden verursachen. Unternehmen können ihren Wettbewerbsvorteil verlieren, ihr Image beeinträchtigen, Umsatzeinbußen erleiden oder sogar Geldstrafen und Sanktionen erhalten.

Um das Risiko von Datenschutzverletzungen zu minimieren, werden immer häufiger Data-Leakage-Prevention (DLP) Lösungen eingesetzt. Gartner prognostiziert, dass bis 2027 etwa 70\% der größeren Unternehmen eine DLP-Lösung einsetzen werden, um die Datensicherheit vor Insider-Risiken und externen Angreifern zu schützen \cite{Chugh.2023}. DLP-Systeme überwachen den Zugriff und Austausch vertraulicher Daten, um unbefugte Offenlegung oder missbräuchliche Nutzung zu erkennen.

% Fortschritt in KI hilft auch Cloud Security
Unternehmen sammeln häufig große Datenmengen, ohne zu wissen, was erfasst wird oder wie sie nach personenbezogenen Daten suchen oder diese abrufen können. Das erschwert den Schutz der Privatsphäre. DLP-Systeme benötigen die Information, ob bestimmte Daten besonders schützenswert sind oder nicht. Im Zeitalter von Big Data ist es jedoch kaum noch möglich, die enormen Datenmengen manuell zu analysieren. Der Fortschritt im Bereich künstliche Intelligenz (KI) bietet hierbei einen vielversprechenden Ansatz. KI-basierte Methoden zur automatischen Datenklassifizierung können in DLP-Systemen eingesetzt werden, um sensible Informationen zu erkennen.

% Fokus der Arbeit, Struktur
Aufgrund der neuen Möglichkeiten mit dem Einsatz von KI im Bereich Datenschutz liegt der Fokus in dieser Arbeit auf der Anwendung von Methoden der automatischen Datenklassifizierung zur Erkennung sensibler Informationen, um den Schutz sensibler Daten zu gewährleisten. Diese Arbeit beschäftigt sich mit der Frage, wie sensible Daten in große Datenmengen am besten erkannt werden können. Dabei wird zunächst die Bedrohung durch versehentliche Offenlegung von Daten beschrieben und anschließend die Abwehrmaßnahme 'Data Leakage Prevention' vorgestellt. Dabei liegt der Fokus auf der Erkennung von sensiblen Informationen. Es werden verschiedene KI-basierte Methoden und ihre Funktionsweise im Bezug auf Datenklassifizierung vorgestellt. Anschließend wird deren Einsatz in der Cloud Sicherheit diskutiert.

\section{Accidental Cloud Data Disclosure} \label{threat-kapitel}
% Einführung der Threats
Die Cloud Security Alliance (CSA) veröffentlicht jährlich einen Bericht über die größten Bedrohungen der Cloud Security, der auf der Befragung von über 700 Experten basiert. Ziel dieses Berichts ist es, auf Bedrohungen, Risiken und Schwachstellen in der Cloud aufmerksam zu machen. Der aktuellste Bericht von 2022 hebt hervor, dass die Verantwortung für die Sicherheit in der Cloud vermehrt vom Cloud Service Provider zum Cloud-Kunden verlagert wird \cite{CloudSecurityAlliance.2022}. % S. 6-8
Diese Verschiebung erhöht das Risiko von Fehlern aufgrund von Unwissenheit. Ein zentrales Sicherheitsproblem ist die unbeabsichtigte Offenlegung von Cloud-Daten ('Accidental Cloud Data Disclosure').

% Accidental Cloud Data Disclosure 
Die versehentliche Offenlegung von Cloud-Daten ist eine Sicherheitsbedrohung, bei der sensible Informationen unbeabsichtigt öffentlich zugänglich gemacht werden. Das kann durch menschliches Versagen, Konfigurationsfehler oder unzureichende Sicherheitsmaßnahmen verursacht werden \cite{CloudSecurityAlliance.2022}. % S. 33-34
% Beispiele
Ein Beispiel ist der Fall von Toyota Motor im Jahr 2023, bei dem persönlichen Daten von Kunden über mehrere Jahre offengelegt wurden. Der Grund war eine Fehlkonfiguration, wodurch die Datenbank in der Cloud öffentlich zugänglich war \cite{Whittaker.2023}.
Ein weiteres Ereignis im August 2023 betraf die nordirische Polizei, als ein interner Mitarbeiter versehentlich persönliche Daten der aktuellen Beamten auf einer Online-Plattform veröffentlichte, indem er die Datei verwechselte \cite{PSNI.2023}.

% Angriffe
Die beiden genannten Beispiele verdeutlichen die potenziell schwerwiegenden Folgen von Fehlern, die schnell zu erheblichem Schaden führen können. Mit der steigenden Verantwortung für die Sicherheit in der Cloud seitens der Kunden erhöht sich das Risiko menschlichen Versagens. Social Engineering und Phishing-Attacken stellen eine Gefahr dar, da Mitarbeiter unbeabsichtigt sensible Daten wie Zugangsdaten offenlegen können. Wie im Fall der nordirischen Polizei gezeigt, besteht auch das Risiko, dass Mitarbeiter Daten unwissentlich veröffentlichen. Zu einer ungewollten Offenlegung von Daten kann es auch kommen, wenn Geräte wie Laptops oder Smartphones, die nicht ausreichend geschützt sind, verloren gehen. Der einfache Zugang zu Cloud-Ressourcen kann dazu führen, dass neue Ressourcen oder Dienste ohne ausreichende Sicherheitsüberlegungen genutzt werden, wodurch Daten aufgrund von Fehlkonfigurationen offengelegt werden könnten. Nicht nur menschliches Versagen, sondern auch Schwächen im Zielsystem stellen Risiken dar. Schwache Passwörter, mangelnde Authentifizierung bei sicherheitsrelevanten Systemen und andere Konfigurationsfehler können bewirken, dass Daten in der Cloud unbeabsichtigt offengelegt werden. Ebenso stellen ungeschlossene Sicherheitslücken in genutzten Cloud-Services eine Bedrohung dar \cite{Trabelsi.2019}\cite{Brindha.2015}.

% Abwehrmaßnahmen
Um das Risiko einer versehentlichen Datenpreisgabe zu minimieren, können verschiedene Schutzmaßnahmen ergriffen werden. Ein kontrolliertes Identity Access Management (IAM) ermöglicht die Regulierung und Kontrolle des internen und externen Datenzugriffs. Die Einführung strenger Passwortrichtlinien und die Nutzung von Passwort-Manager-Software reduzieren das Risiko unbefugten Zugriffs auf Geräte, Benutzerkonten oder Cloud-Ressourcen. Das Prinzip des geringsten Privilegs gewährleistet, dass Benutzer nur die notwendigen Berechtigungen für ihre Aufgaben erhalten, was das Risiko von Fehlkonfigurationen oder missbräuchlichem Zugriff minimiert. Neben der Kontrolle der Zugriffe ist auch die Überwachung möglicher Schwachstellen entscheidend. Regelmäßige Schwachstellen-Scans helfen, Sicherheitslücken in der Cloud-Infrastruktur zu identifizieren und zu beheben, bevor sie ausgenutzt werden können. Die Überprüfung und Optimierung von Cloud-Konfigurationen gewährleistet korrekte Sicherheitseinstellungen. Eine zentrale Verwaltung aller in der Cloud vorhandenen Assets ermöglicht eine bessere Kontrolle und Überwachung von Daten, Diensten und Einstellungen. Regelmäßige Softwareaktualisierungen sind wichtig, um bekannte Sicherheitslücken zu schließen. Mitarbeiter sollten zudem durch Schulungen für sicherheitsrelevante Themen sensibilisiert werden, um menschliche Fehler zu minimieren \cite{Brindha.2015}.

% Überleitung Cloud Data Leakage Prevention
Im Kontext dieser Bedrohung werden die Begriffe Data Loss (Datenverlust) und Data Leakage (Datenleck) häufig als Synonyme verwendet, weisen jedoch einige Unterschiede auf. Datenverlust bezieht sich auf den unwiederbringlichen Verlust von Daten, beispielsweise durch Schäden an Speichermedien, unbeabsichtigtes Löschen oder Hardwarefehler. Im Gegensatz dazu bezeichnet Datenleck die unbeabsichtigte oder absichtliche Übertragung von Daten aus einem gesicherten Bereich. Datenlecks können auftreten, wenn unbefugte Personen sensible oder vertrauliche Informationen erhalten \cite{Proofpoint.2021b}. Daher wird in dieser Arbeit der Begriff Datenleck oder Data Leakage verwendet, um die unbeabsichtigte Offenlegung von Daten zu beschreiben.

Die zunehmende Komplexität der Cloud-Infrastrukturen und die steigende Verantwortung der Kunden für die Sicherheit haben das Risiko der versehentlichen Offenlegung sensibler Daten erheblich erhöht. In Anbetracht dieser Herausforderungen und um das Risiko von Datenlecks zu minimieren, werden DLP-Systeme zunehmend als entscheidende Sicherheitsmaßnahme eingesetzt.




\section{Cloud Data Leakage Prevention System}
% Aufnahme Data Leakage Prevention in ISO
Im vorherigen Kapitel \ref{threat-kapitel} über die Bedrohung durch versehentliche Datenoffenlegung wurden deutlich, wie schnell ein Datenleck in Unternehmen auftreten kann. Die Bedeutung effektiver Maßnahmen zur Vermeidung von Datenlecks hat sich aufgrund der wachsenden Datenmengen und des damit verbundenen Risikos einer Datenschutzverletzung erhöht. Dieser Bedarf wurde 2022 erkannt, als in der neuesten Version der Norm ISO 27001:2022 die Data Leakage Prevention eingeführt wurde. Die internationale Norm ISO 27001 definiert die Bedingungen für die Einrichtung, Umsetzung und kontinuierliche Verbesserung eines dokumentierten Informationssicherheits-Managementsystems. Die Norm gibt außerdem Vorschriften für die Beurteilung und Behandlung von Informationssicherheitsrisiken, die an die spezifischen Bedürfnisse jedes Unternehmens angepasst werden müssen \cite{Monev.2023}.
% DLP Abkürzung vorher schon eingeführt?
Ein Datenleck kann auf verschiedene Weise auftreten. Trotz der Tatsache, dass es nicht immer möglich ist, das Auftreten vollständig zu verhindern, können Maßnahmen ergriffen werden, um die Wahrscheinlichkeit eines Auftretens zu verringern. Diese Maßnahmen werden als Data Leakage Prevention (DLP) bezeichnet \cite{Monev.2023}. Dabei handelt es sich um eine Reihe von Technologien, Produkten und Methoden, die dazu dienen, zu verhindern, dass vertrauliche Informationen ein Unternehmen verlassen. In den letzten Jahrzehnten wurden verschiedene Sicherheitssysteme wie Firewalls, Intrusion-Detection-Systeme (Einbrucherkennung) und virtuelle private Netzwerke (VPN) eingeführt, um das Risiko von Datenlecks zu reduzieren. Wenn die zu schützenden Daten klar definiert, strukturiert und konstant sind, erfüllen diese Systeme ihren Zweck. Jedoch sind sie unzuverlässig für Daten, die sich ändern oder unstrukturiert sind. Durch einfache Regeln kann beispielsweise eine Firewall den Zugriff auf ein sensibles Datenobjekt verhindern. Die Firewall erkennt jedoch nicht, wenn das Datenobjekt über einen E-Mail-Anhang gesendet wird. DLP-Systeme hingegen sind darauf spezialisiert, vertrauliche Daten zu identifizieren, zu überwachen und zu schützen und unerwünschte Datenbewegungen zu verhindern.\cite{Alneyadi.2016}.

\subsection{Cloud Data Leakage Prevention System}
Ein DLP-System umfasst eine Reihe von Regeln und Richtlinien, die Daten nach ihrem Typ klassifizieren, um sicherzustellen, dass sie nicht böswillig oder versehentlich weitergegeben werden. Das System überwacht Endbenutzeraktivitäten, den Datenfluss sowie die über das Netzwerk gesendeten Daten. Wenn verdächtige Aktivitäten erkannt werden, wird eine Systemwarnung ausgelöst. DLP-Lösungen identifizieren sensible Inhalte mithilfe von Datenklassifizierung-Label, Techniken zur Inspektion von Inhalten und Kontextanalysen. Sie überwachen die Datenaktivität und kontrollieren sie anhand vordefinierter DLP-Richtlinien. Die Richtlinien definieren, ob die Verwendung bestimmter Inhalte oder Daten in bestimmten Situationen erlaubt sind \cite{Chugh.2023}.

Gartner klassifiziert DLP-Lösungen in drei Kategorien. Eine Enterprise-DLP-Lösung ist ein zentrales System, das darauf ausgelegt ist, komplexe Anforderungen und Strukturen großer Unternehmen zu bewältigen. Sie verfügt über fortschrittliche Technologien zur Identifikation, Klassifizierung und Markierung sensibler Daten und ist in der Lage, verschiedene Datenquellen zu integrieren. So kann diese Lösung den gesamten Lebenszyklus von Daten in einem Unternehmen abdecken. DLP-Richtlinien werden dabei an zentraler Stelle verwaltet und durchgesetzt.
Dagegen werden integrierte DLP-Lösungen direkt in einen Dienst, wie bspw. ein E-Mail-Gateway, integriert und verfügen deshalb nur über begrenzte Richtlinienfunktionen. Das Management von mehreren integrierten DLP-Systemen ist ein manueller Aufwand, jedoch werden diese Systeme im jeweiligen Dienst speziell an die Anforderungen angepasst und können Inhaltsüberprüfungen besser durchführen.
Cloud-native DLP-Lösungen sind die dritte Kategorie, zu der sowohl SaaS-Lösungen als auch Cloud-Anbieter mit integrierten DLP-Funktionen gehören. Sie sind speziell für den Einsatz in Cloud-Umgebungen entwickelt und darauf ausgerichtet, sensible Daten in Cloud-Diensten zu schützen. Diese Lösungen verfügen über Mechanismen zur automatischen Erkennung von sensiblen Daten, die in Cloud-Anwendungen und -Speicherplätzen gespeichert sind. Dies umfasst die Identifikation von Daten in Form von Dokumenten, E-Mails, Datenbanken und anderen Formaten \cite{Chugh.2023}.
Im weiteren Verlauf der Arbeit wird der Begriff DLP-System für alle drei Kategorien verwendet.

% Aufbau / Funktionen
% Bild von NIST oder DLP Funktionen einfügen?
Das Cybersecurity Framework des National Institute of Standards and Technology (NIST CSF) bietet freiwillige Standards und Best Practices, die Unternehmen dabei helfen, Cybersecurity-Risiken zu managen und zu reduzieren. Es gibt Unternehmen eine Struktur, um ihre aktuelle Cybersicherheitssituation zu bewerten, verbesserungsbedürftige Bereiche zu identifizieren, Maßnahmen zu priorisieren, Fortschritte zu bewerten und mit den Stakeholdern zu kommunizieren. Die CSF besteht aus fünf Kernfunktionen: Identifizieren, Schützen, Erkennen, Reagieren und Wiederherstellen.
DLP-Systeme konzentrieren sich hauptsächlich auf die Identifizierung, die Erkennung und den Schutz und ergänzen diese Funktionen durch den Bereich der Überwachung. Die spezifischen Funktionen eines DLP-Systems können je nach Hersteller variieren \cite{NIST.2014}.

% TODO: Literatur-Recherche  + Ergebnisse als Tabelle darstellen?
Die Literatur-Recherche ergab die folgende Auswahl an Best-Practises, die in DLP-Systemen eingesetzt werden sollten. Um sensible Daten schützen zu können, müssen diese zuerst identifiziert werden. Die Aufgabe besteht darin, ein Dateninventar zu erstellen, die Daten nach ihrer Sensibilität zu klassifizieren und sie entsprechend zu kennzeichnen. Zum Schutz der sensiblen Daten sollten Maßnahmen ergriffen werden, die den Zugriff auf die Daten einschränken. Das bedeutet, dass Richtlinien wie minimale Zugriffsrechte, starke Authentifizierungsmethoden und strenge Zugriffskontrolllisten eingeführt werden sollten. Außerdem sollten Daten sowohl im Ruhezustand als auch während der Übertragung verschlüsselt werden. So wird sichergestellt, dass die Daten selbst dann, wenn sie abgefangen werden, für unbefugte Benutzer unlesbar bleiben. Zusätzlich sollte ein DLP-System die Datenströme innerhalb und nach außen überwachen, um potenzielle Datenschutzverletzungen oder Richtlinienverstöße in Echtzeit erkennen zu können. Dies ermöglicht eine schnelle Reaktion auf potenzielle Probleme und begrenzt den daraus resultierenden Schaden \cite{Hussain.2022}\cite{HerreraMontano.2022}\cite{Shishodia.2022}. % es gibt noch weitere Möglichkeiten?

% Herausforderung Daten Klassifizierung
Die Funktionen eines DLP-Systems basieren alle darauf, dass sensible Daten erkannt und in irgendeiner Art markiert sind. Der erste Schritt bei DLP-Systemen ist daher die Identifizierung sensibler Daten. Es gibt verschiedene Strategien und Methoden zur Klassifizierung dieser Daten, die durch den Einsatz von KI weiter verbessert wurden.

\subsection{Erkennung von sensiblen Daten}
% Warum Datenklassifizierung? 
% TODO: SaaS schon eingeführt?
Unternehmen setzen immer mehr auf SaaS-Produkte, anstatt sie als Produkt zu kaufen \cite{Gartner.2023}. In ihrem Tagesgeschäft verlassen sich Unternehmen oft auf mehrere Softwareprodukte, um verschiedene Anforderungen zu erfüllen. Das hat zur Folge, dass die Daten des Unternehmens über verschiedene Apps und Cloud-Plattformen verstreut sind. Die Herausforderung besteht darin, den Überblick zu behalten und zu wissen, wo sich die sensiblen Daten befinden. Das Sammeln und Identifizieren von Daten in DLP-Systemen stellt aufgrund von Verschlüsselung, verborgenen Kanälen, nicht unterstützten Datenformaten und großer Mengen an Daten eine große Herausforderung dar \cite{Hauer.2015}.

Die Methoden zur Erkennung und Klassifizierung von sensiblen Daten unterscheiden sich je nach Art und Format der Daten, sowie deren Zustand. Außerdem gibt es die Möglichkeit, Daten manuell oder automatisiert zu klassifizieren.

\subsubsection{Eigenschaften von Daten}
% Art -> welche Kategorien
Sensible Daten durchdringen fast jeden Aspekt unseres persönlichen und beruflichen Lebens. Das Spektrum dieser sensiblen Informationen reicht von persönlichen Daten über Finanzinformationen und Geschäftsgeheimnisse bis hin zu biometrischen Merkmalen und umfasst eine Vielzahl von Kategorien. Guo, Liu et al. \cite{Guo.2021} kategorisiert beispielsweise Daten in die vier Bereiche:
\begin{itemize}
    \item Persönliche Informationen (z.B. Name, Geburtsdatum oder Gesundheitsinformationen)
    \item Informationen zur Netzwerkidentität (z.B. IP-Adresse, MAC-Adresse oder E-Mail)
    \item Vertrauliche und Anmeldeinformationen (z.B. Login-Passwort-Kombinationen, API-Token oder digitale Zertifikate)
    \item Finanzinformationen (z.B. Bankkontodaten, Kreditkarteninformationen oder Verbrauchsdaten)
\end{itemize}
Die Kategorisierung und der Detailgrad können je nach Unternehmen variieren.

Neben den Kategorien muss auch der Kontext beachtet werden, in dem eine Information verwendet wird. Denn der Kontext hat direkten Einfluss auf die Sensibilität. Pogiatzis und Samakovitis \cite{Pogiatzis.2020} leiten vier verschiedene Kontextklassen ab, die auf der Bedeutung, der Interaktion, der Priorität und Präferenz basieren, die mit jeder Information verbunden sind.

\begin{itemize}
    \item Der semantische Kontext wird auf Grundlage der semantischen Bedeutung eines Begriffs gebildet. Die semantische Bedeutung einer Sequenz wirkt sich zum Beispiel auf ihre Sensibilität aus.
    \item Im Kontext der Akteure wird die Sensibilität von Daten abhängig von den Akteuren, die an der Informationsübermittlung beteiligt sind, betrachtet. Die Sensibilität wird durch die Beziehung zwischen den beteiligten Akteuren bestimmt. Zum Beispiel ist der Austausch von Gesundheitsinformationen zwischen Patient und Arzt nicht sensibel, außerhalb dieser Gruppe von Akteuren jedoch schon.
    \item Der zeitliche Kontext bezieht sich auf die Priorität der Informationen, die die Bedeutung des Begriffs beeinflussen. Eine Zeichenfolge, die als Passwort eingegeben wird, gilt bspw. als vertraulicher, als wenn sie als Benutzername eingegeben wird.
    \item Der Selbstkontext wird durch die persönlichen Präferenzen des Nutzers in Bezug auf seine Privatsphäre bestimmt. Zum Beispiel kann eine Person ihre ethnische Herkunft als vertrauliche Information betrachten, eine andere nicht.
\end{itemize}
Auch hier können verschiedene kontextuelle Kategorien unterschieden oder definieren werden. Sie sind zudem nicht immer klar trennbar und schließen sich nicht gegenseitig aus. Ein oder mehrere Kontexte können sich gleichzeitig unterschiedlich auf die Sensibilität auswirken. Manche Daten können jedoch auch unabhängig vom Kontext vertraulich sein, wie z.B. Passwörter oder Kreditkartennummern.

% Daten Klassen
Die Einteilung von Daten in verschiedene Geheimhaltungsklassen ist ein häufig verwendetes Verfahren in militärischen und behördlichen Anwendungen. Militärische Anwendungen verwenden dabei Begriffe wie \glqq eingeschränkt\grqq , \glqq vertraulich\grqq, \glqq geheim\grqq und \glqq streng geheim\grqq \cite{Landwehr.1984}. So ist es möglich, sensible Daten noch präziser in Vertraulichkeitsstufen zu unterteilen.

% Format -> structured, unstructured
Außerdem wird die Klassifizierung auch von der Struktur der Daten beeinflusst. Mehr als 80\% der Daten im Internet bestehen aus unstrukturierten Daten \cite{Allahyari.2017}. Unstrukturierte Daten beziehen sich in der Regel auf Informationen, die nicht in einer relationalen Datenbank gespeichert sind. Folglich gibt es kein vordefiniertes Datenmodell und die Struktur ist unregelmäßig oder unvollständig. Selbst Datenformate wie CSV, JSON oder XML, die einige organisatorische Eigenschaften haben, verfügen in der Regel nicht über ein klar definiertes Datenmodell. Im Vergleich zu strukturierten Daten ist es schwieriger, unstrukturierte Daten abzurufen, zu analysieren und zu speichern. Während Menschen unstrukturierte Daten leicht verarbeiten können, haben Maschinen oft Schwierigkeiten damit \cite{Guo.2021}.

Die Herausforderung bei der Datenklassifizierung besteht daher darin, die Kategorien, den Kontext, die Vertraulichkeitsstufen und die Struktur der Daten zu berücksichtigen.

\subsubsection{Daten Zustand}
% Stati von Daten (in Rest, in Motion etc.)

\subsubsection{Manuelle Datenklassifizierung}
% Warum manuell

% warum schwierig (aufwändig, fehlerbehaftet, Big Data)

\subsubsection{Automatische Datenklassifizierung}
% warum automatisiert

% wie automatisiert

% content vs. context statt rule/model/context

\paragraph{content-based}

\paragraph{context-based}



\subsection{Anwendung in der Cloud Security}
% wofür ist das dann gut

\subsubsection{Labeling für andere Maßnahmen}
% verschlüsselung
% Access Rights
% Versenden

\subsubsection{Anwendungsbeispiele}
% BrowserFlow
% DocGuard?

\section{Ausblick}
% Stärken und Schwächen

% Verbesserung im Kampf

% references ieeetr
\balance
\bibliographystyle{IEEEtran}
\bibliography{ref}

\end{document}
